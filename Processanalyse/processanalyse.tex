\documentclass[a4paper,11pt,fleqn,oneside,openright]{memoir}

\usepackage{fourier}   %font

\usepackage[utf8]{inputenc}					% Gør det muligt at bruge æ, ø og å i sine .tex-filer
\usepackage[danish]{babel}							% Dansk sporg, f.eks. tabel, figur og kapitel
\usepackage[T1]{fontenc}

\begin{document}
% Skriv her
\title{Processanalyse}
\author{B2-1a}
\date{23. september 2013}
\maketitle

\section{beskrivelse}

\subsection{Projektplanlægning}
\begin{itemize}
\item Enighed om standarder
\item Brainstorm af ideer
\subitem Frste forslag
\item Tidsplan-overordnet
\subitem Mundlige deadlines
\item Logbog-til at starte med
\item Fase2 Brainstorm af problemer
\end{itemize}

\subsection{Gruppesamarbejde}

\begin{itemize}
\item Gruppe-kontrakt
\item Social aktivitet
\item Hvordan  foregik arbejdsfordeling
\item Uddeling af opgaver
\item Interesse=inddeling, derefter kompitance
\end{itemize}

\subsection{samarbejde med vejlederen}

\begin{itemize}
\item Aftalte møde til næste gang
\item Forberedelse til møde
\item Forsøgte at sende feedback

Før hvert møde, havde vi forberedt spørgsmål til vejlederen, og punkter vi gerne ville have respons på. Det var bl.a. forvirring angående rapportens opstilling og krav til rapportens generelle indhold. Dertil kom spørgsmålet om vi var på rette kurs.

Vi sørgede for at aftale næste møde med vejlederen, således at vi havde en ny "deadline" for spørgsmål. Vi aftalte med vejlederen at sende arbejdsblade/foreløbigt arbejde via email.

\end{itemize}

\subsection{udarbejdelse af problemformulering}
\begin{itemize}
\item Iteretion af mindmaps
\item Vejleder hjælp, hjalp
\item Løbende ændringer af problemformuleringer

\end{itemize}

\subsection{rapportstrukturering}
\begin{itemize}
\item Halvejs inde – rød tråd (sofa \& post-it)
\item Inspiration af andre rapporter
\item Mindmaps (overskrifter)
\item Selvkritik(mangel)
\end{itemize}

Rapportstruktureringen har i gennem hele perioden været under mange ændringer, grundet selvkritik og kritik fra vejlederen.

Vi dannede vores første rapportstrukturering ud fra mindmaps. Vi kom frem til 3 hovedemner, som dannede grundlag for rapporten. Senere efter samtale med vejleder, lavede vi en ny struktur der præciserede rapportstrukturen. Dette blev gjort ud fra idéen om at skrive om tidligere, nuværende og fremtidige forhold. Efter endnu en samtale med vejleder, satte vi post-it's på tavlen med hensigt på at finde en struktur der havde en rød tråd. Den tredje strukturering var mere detaljeret, og tog betydelig mere tid.

Hele vejen igennem rapportstruktureringen, trak vi på inspiration fra andre rapporter.

\subsection{Andet}
\begin{itemize}

\item Selvsving (cloudslicing) 
\end{itemize}

Vi havde tendens til at spore os ind på én bestemt tangent. F.eks. begynde vi for tidligt at tænke på løsningsforslaget, inden vi overhovedet havde defineret et egentligt problem.



\section{Vurdering - eller Hvordan gik det?}



\section{Analyse - eller Hvorfor gik det som det gik?}



\section{Syntese - eller gode råd til p1}








% chapter test (end)
\end{document}