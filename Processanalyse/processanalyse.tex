\documentclass[a4paper,11pt,fleqn,oneside,openright]{memoir}

\usepackage{fourier}   %font

\usepackage[utf8]{inputenc}					% Gør det muligt at bruge æ, ø og å i sine .tex-filer
\usepackage[danish]{babel}							% Dansk sporg, f.eks. tabel, figur og kapitel
\usepackage[T1]{fontenc}

\begin{document}
% Skriv her
\title{Processanalyse}
\author{B2-1a}
\date{23. september 2013}
\maketitle



\section{beskrivelse}

\subsection{Projektplanlægning}

Vores projekt startede med en idegenerering i form af en brainstorm. Vi startede med at gennemgå de udleverede projektforslag. Hvorefter vi valgte at opstille egne problemstillinger. Vi valgte at bruge en af de problemstillinger som vil selv opstillede. 
Vi valgte derefter hvilke programmer og værktøjer, som vi ville bruge i løbet af projektet. Dette gjorde vi for beder at kunne strukturer opgaven.  Der blev aftalt hvilket kommunikationsprogram der skulle bruges, så alle i gruppen vidste hvordan man nemt kunne opsøge hinanden. 
Vi lavede en tidsplan på første dag for at skabe et overblik over projektforløbet, og for at sætte interne deadlines til deleafleveringer. Tidsplanen blev tilpasset et stykke inde i forløbet da vi vidste mere om projektet.
Vi lavede en logbog der skulle have til formål at hjælpe os med at holde styr på aftaler og de fremskridt vi gjorte. Logbogen blev opdateret på dag-til-dag’s basis hvilket gjorte at der ikke var tvivl om hvad der skulle være klar til de følgende dage. Logbogen redegjorte også for hvad der blev snakket om under vejledermøderne. 

\subsection{Gruppesamarbejde}

Vi startede med at lave en gruppe-kontrakt, som omfatter mødetid, arbejdssteder, logbogsskrivning, konflikthåndtering, sygdomsforløb, tidsplanskrav, vejledermøder og krav om morgenmøde.

Udenfor projektarbejdet, deltog gruppemedlemmer i forskellige sociale aktiviteter som spisning, knoldbold mm.

Når vi opdeler arbejdet, stræbes der efter arbejde 2 og 2. Vi har uddelt først og fremmest efter interesse, dernæst kompetence.

\subsection{Samarbejde med vejlederen}

Før hvert møde, havde vi forberedt spørgsmål til vejlederen, og punkter vi gerne ville have respons på. Det var bl.a. forvirring angående rapportens opstilling og krav til rapportens generelle indhold. Dertil kom spørgsmålet om vi var på rette kurs.

Vi sørgede for at aftale næste møde med vejlederen, således at vi havde en ny "deadline" for spørgsmål. Vi aftalte med vejlederen at sende arbejdsblade/foreløbigt arbejde via email.

\subsection{Udarbejdelse af problemformulering}

Vi starte med at lave et mindmap med eventuelle emner, 3D-print, ParkeringsApp og CykelApp. Derfra valgte vi 3d-print, dertil et nyt mindmap med problemstilling, Jura, Tekniske ting, forretningsmodeller og interesser. En iteration af mindmaps mere, opstod Cloudslicing og salg af schematics. Ved hjælp at disse iterationer fik sporet os ind på en vej, vi ville tage fat i.

Vi præsenterede en først et mindmap med salg af schematics, og fik respons på vores ideer. Dernæst en foreløbig problemformulering til vejlederen, og tog hans respons til overvejelse. Til sidst fik vejlederen et udkast af rapporten og den struktur.

Undervejs i forløbet blev vores problemformulering ændret, efterhånden som vi udvidet vores viden om emnet og dermed kunne lave afgrænsninger.

\subsection{Rapportstrukturering}

Rapportstruktureringen har i gennem hele perioden været under mange ændringer, grundet selvkritik og kritik fra vejlederen.

Vi dannede vores første rapportstrukturering ud fra mindmaps. Vi kom frem til 3 hovedemner, som dannede grundlag for rapporten. Senere efter samtale med vejleder, lavede vi en ny struktur der præciserede rapportstrukturen. Dette blev gjort ud fra idéen om at skrive om tidligere, nuværende og fremtidige forhold. Efter endnu en samtale med vejleder, satte vi post-it's på tavlen med hensigt på at finde en struktur der havde en rød tråd. Den tredje strukturering var mere detaljeret, og tog betydelig mere tid.

Hele vejen igennem rapportstruktureringen, trak vi på inspiration fra andre rapporter.

\subsection{Andet}

Vi havde tendens til at spore os ind på én bestemt tangent. F.eks. begynde vi for tidligt at tænke på løsningsforslaget, inden vi overhovedet havde defineret et egentligt problem.

\section{Vurdering - eller Hvordan gik det?}

\subsection{Projektplanlægning}

Tidsplanen var utilstrækkelig, og blev derfor ikke brugt. Det var godt at vi nåede til enighed i brugen af Skype, LaTeX og GitHub.

Vi startede ud med at lave logbog over hvad vi har udrettet, og hvad vi skal lave den næste dag. Men vi fik ikke opdateret den.

\subsection{Gruppesamarbejde}

Idéen med gruppe-kontrakten var udemærket, men udformningen var ikke helt færdig. Vores krav til hinanden var ikke helt realistiske.

Gruppen var god til at deltage i sociale aktiviteter, som hjalp med at styrke båndene mellem gruppens medlemmer.

2 og 2 arbejdet var en succes, da man kunne få input, hurtigt.


\subsection{Samarbejde med vejlederen}

Det var godt vi løbende aftalte møder. Når vi havde forberedt spørgsmålene (som blev skrevet ned når de kom), var de relevante. Vi tog godt i mod respons fra vejleder. Vores aftaler omkring respons af arbejdsblade, var utilstrækkelig, da vi sendte halvfærdige, ikke gennemrettede arbejdsblade til ham.

\subsection{Udarbejdelse af problemformulering}



\subsection{Rapportstrukturering}

\subsection{Andet}

\section{Analyse - eller Hvorfor gik det som det gik?}



\section{Syntese - eller gode råd til p1}








% chapter test (end)
\end{document}