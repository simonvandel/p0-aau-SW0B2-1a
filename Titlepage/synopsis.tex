\chapter{Synopsis} % (fold)
\label{cha:synopsis}


\subsection{Problemstilling} % (fold)
\label{sub:problemstilling}

Manglende muligheder for køb af schematics til private printere - hæmmer udviklingen indenfor udvikling af 3D-printer og 3D-schematics-teknologien.

3D-printerens potentiale til at kunne benyttes af den almene befolkning, hindres pga. tilgængelighed da prisen for en 3D-printer ligger langt ud over hvad den gennemsnitlige borger vil betale.

Manglende opmærksomhed fra medierne, mindsker interessen for store elektronikvarehuse som f.eks. Elgiganten

% subsection problemstilling (end)


\subsection{Problemanalyse} % (fold)
\label{sub:problemanalyse}

Ingen populære hardwarebutikker har en visuelt udstilling af 3D-printer.

For lidt konkurrence og for lidt opmærksomhed omkring køb af 3D-printere på danske online-webshops.

Potentiel piratkopiering af 3D-schematics, skræmmer store designfirmaer til at indgå i markedet.

Mange gråzoner indenfor ophavsret og ulovlige printede produkter, gør markedet uoverskueligt.

Manglende standartisering indenfor 3D-printer udskrivning, gør det besværligt for både designere, sælgere af 3D-schematics og forbrugere.

% subsection problemanalyse (end)


\subsection{Problemformulering} % (fold)
\label{sub:problemformulering}

Belyse tekniske problemstillinger ved distribution af 3D-schematics.

% subsection problemformulering (end)




\subsection{Relevant teori} % (fold)
\label{sub:relevant_teori}

Indsigt i eksisterende viden omkring 3D-printing, f.eks. forbindelse fra computer til printer.

Lovgivning omhandlende ophavsret.

\subsection{Ny Synopsis}
\label{sub: Synopsis}
Dette project lægger vægt på de problemer og udfordringer, der er ved det juridiske- og tenologiske aspekt af 3D-printning, herunder lægges der også vægt på perspektivering til andre problemer ved lignende ting. Projectet kommer ind på, den nuværende og fremtidige teknologi, samt omfanget af 3D-printnig i Danmark.


% subsection relevant_teori (end)
% chapter synopsis (end)
