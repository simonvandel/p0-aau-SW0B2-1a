\chapter{Introduktion} % (fold)
\label{cha:introduktion}

I mange år har man kendt til ideen fra Science fiction film. I filmen "Hitchhikers Guide to the Galaxy", kan en køkkenmaskine læse dine tanker og lave den ret, du allermest ønsker dig. Der findes udtallige eksempler, men de har alle det samme til fælles; De får ting til at fremstå ud af den tilsyneladende blå luft.

Men nu behøver vi ikke længere, drømme os ind i Science Fiction; 3D-printeren er kommer for at blive. Imodsætning til science fiction, er vi desværre begrænset af fysikens love, men man kan næsten blive snydt, når man ser, en maskine fremstille diverse plastic dimser med den fineste præcision.

De teknologiske fremskridt bliver større og større, for hver dag der går. Alle eksperter spår 3D-printerer en stor fremtid, nogle mener sågar, at det bliver lige så stort som internettet. Lægevidenskaben har formået at printe visse organer, som hud og selv et funktionelt hjerte.

Det lader til at 3D-printerens muligheder ingen ende vil tage. Men hvorfor har vi så ikke alle en stående i vores stuer? Eller i det mindste en man deler med vennerne? Hvorfor er 3D-printeren ikke blevet hvermandseje? Dette virker yderst mærkværdigt. 

I den efterfølgende rapport, vil der blive kigget nærmere på netop dette problem. Vi vil forsøge at beskrive de mange facetter af problemet, og kortlægge de områder hvor der er behov for løsninger.

% chapter introduktion (end)