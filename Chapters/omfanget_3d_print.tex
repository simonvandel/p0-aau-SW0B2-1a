%!TEX root = ../Master.tex
\section{Omfanget af 3D-printning} % (fold)



\subsection{Nuværende platforme} % (fold)


\paragraph{3D Burrito} % (fold)
\label{par:3d_burrito}

En anden markedsplads for 3D-printning, er 3D Burrito \autocite{_3d_2013-1}. Her formidler 3D Burrito STL-filerne til køberne. Som køber får du altså den digitale 3D-model. Det er så op til dig hvad du vil gøre med den. Der er mulighed for at ændre i STL filen, og på den måde ændre modellen. Der er også mulighed for at printe STL filen på en personlig 3D-printer.

% paragraph 3d_burrito (end)

\paragraph{Thingiverse} % (fold)
\label{par:thingiverse}

En kæmpe brugergenereret markedsplads for 3D-schematics er MakerBots Thingiverse. Her kan man frit downloade 3D-schematics. I juni 2013 rundede Thingiverse 100.000 things, som de kalder de oploadede 3D-schematics. \autocite{_100000th_????}

% paragraph thingiverse (end)

\paragraph{Problemer/mangler} % (fold)
\label{par:problemer_mangler}

Når man får fat i et 3D-schematic er der stadig en process af slicing og konfigurering af printeren, der skal gøres, før man kan begynde at printe. Denne process er i sin nuværende tilstand for kompliceret til at hr. og fru Jensen uden videre kan 3D-printe.

% paragraph problemer_mangler (end)

\subsection{Salg af 3D-printere}

Salget for 3D-printere er under stor udvikling \autocite{wohler_sales_2012}. Det estimeres at det globale salg af 3D-printere i 2025 vil vokse til 4 billiarder \$. \autocite[110]{manyika_disruptive_2013}

% \subsection{Fordelene} % (fold)
% \label{sec:fordelene}


% Komplekse produkter kan nemt produceres og med mulighed for stor brugertilpasning. Boeing printer lige nu 200 forskellige dele til 10 flyplatforme. I sundhedssektoren har man i 2011 solgt mere end 1 million tilpassede høreapparater. \footcite[108]{manyika_disruptive_2013}. Små produkter som kan printes direkte fra CD-filen på designerens computer, øger produktiviteten omkring udvikling af nye produkter. Ydermere da printeren har de rå materialer, bliver samleprocessen kortere, og dermed billigere o bedre for miljøet da en masse transportering af underdele undgås.

% % section fordelene (end)

% \subsection{Ulemperne} % (fold)
% \label{par:ulemperne}


% Prisen på 3D-printeren er stadig rimelig dyr, og derfor er det typisk kun hobbyister og generelle nørder der indtil videre har erhvervet sig en 3D-printer. En anden ulempe, er også at det er en tidskrævende process at printe en genstand. Derudover er der nogle begrænsninger på størrelse af objektet der kan printes.


% % section ulemperne (end)

% \subsection{En del af husstanden} % (fold)
% \label{par:en_del_af_husstanden}


% I takt med at disse ulemper begrænses ved teknologisk fremgang, vil 3D-printeren blive en almindelig maskine at have stående hjemme. Man kan forestille sig at frøken Jensen mangler en dugholder til havefesten hun skal holde den kommende dag. Hun går ind på nettet, finder en dugholder hun godt kan lide og trykker derefter på print. Afhængig af printeren og kvaliteten af printet, vil dugholderen være klar til brug under havefesten. Dette lyder måske rimelig sci-fi, men er faktisk allerede muligt. I et kommende afsnit beskriver vi de nuværende 3D-printer services.
% % section en_del_af_husstanden (end)

% % chapter omfanget_af_3d_printer (end)