\chapter{Omfanget af 3D-printning} % (fold)
\label{cha:omfanget_af_3d_printer}

Idet de fleste patenter omhandlende 3D-printning er ved at udløbe, kan teknologien der bruges til 3D-printere for alvor udvikle sig. 3D-printere har eksisteret i industrien i lang tid, hvor de er blevet brugt til f.eks. at lave hurtige prototyper. Nu hvor teknologien er blevet bedre, er 3D-printere blevet hurtigere, mere præcise og også billigere. Dette har gjort det muligt for forbrugere at købe personlige 3D-printere. 


Fordelen ved 3D-printere er at komplekse produkter nemt kan produceres og med mulighed for stor brugertilpasning. Boeing printer lige nu 200 forskellige dele til 10 flyplatforme. I sundhedssektoren har man i 2011 solgt mere end 1 million tilpassede høreapparater. \cite{manyika_disruptive_2013} Små produkter som kan printes direkte fra CAD-filen på designerens computer. Da printeren har de rå materialer, bliver samleprocessen kortere, og dermed billigere og bedre for miljøet da en masse transportering af materialer undgås.


Ulemperne ved 3D-printning lige nu er prisen på 3D-printeren og også at det er en tidskrævende process at printe en genstand. Derudover er der nogle begrænsninger på størrelse af objektet der printes.


I takt med at disse ulemper begrænses ved teknologisk fremgang, vil 3D-printeren blive en almindelig maskine at have stående hjemme.

% chapter omfanget_af_3d_printer (end)