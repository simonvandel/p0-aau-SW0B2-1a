\section{Omfanget af 3D-printning} % (fold)

\subsection{En teknologisk udvikling} % (fold)

Idet de fleste patenter omhandlende 3D-printning er ved at udløbe, kan teknologierne der bruges til 3D-printere for alvor udvikle sig. Sådan en udvikling kunne man f.eks. se da patentet på FDM printer-teknologien udløb i 2009. \autocite{manyika_disruptive_2013} Umiddelbart efter dette FDM patent udløb, kom MakerBot på markedet. MakerBot er en videreudvikling af RepRap, hvor MakerBot forsøger at forbedre slutbrugeroplevelsen. \autocite{_makerbot_2013}

% section en_teknologisk_udvikling (end)

\subsection{Nuværende platforme} % (fold)

En af de eksisterende platforme på markedet er shapeways \autocite{_shapeways_????}. På deres platform kan man vælge et produkt fra brugeroploadede designs, få det printet og derefter sendt hjem til køberen. Prisen på deres varer er dog stadig høj. F.eks. koster en ske af plastik omkring 85 kroner at få sendt til Danmark. \autocite{_magic_????}

\subsection{Salg af 3D-printere}

Salget for 3D-printere er under stor udvikling \autocite{wohler_sales_2012}. Det estimeres at det globale salg af 3D-printere i 2025 vil vokse til 4 billiarder \$. \autocite[110]{manyika_disruptive_2013}

% \subsection{Fordelene} % (fold)
% \label{sec:fordelene}


% Komplekse produkter kan nemt produceres og med mulighed for stor brugertilpasning. Boeing printer lige nu 200 forskellige dele til 10 flyplatforme. I sundhedssektoren har man i 2011 solgt mere end 1 million tilpassede høreapparater. \footcite[108]{manyika_disruptive_2013}. Små produkter som kan printes direkte fra CD-filen på designerens computer, øger produktiviteten omkring udvikling af nye produkter. Ydermere da printeren har de rå materialer, bliver samleprocessen kortere, og dermed billigere o bedre for miljøet da en masse transportering af underdele undgås.

% % section fordelene (end)

% \subsection{Ulemperne} % (fold)
% \label{par:ulemperne}


% Prisen på 3D-printeren er stadig rimelig dyr, og derfor er det typisk kun hobbyister og generelle nørder der indtil videre har erhvervet sig en 3D-printer. En anden ulempe, er også at det er en tidskrævende process at printe en genstand. Derudover er der nogle begrænsninger på størrelse af objektet der kan printes.


% % section ulemperne (end)

% \subsection{En del af husstanden} % (fold)
% \label{par:en_del_af_husstanden}


% I takt med at disse ulemper begrænses ved teknologisk fremgang, vil 3D-printeren blive en almindelig maskine at have stående hjemme. Man kan forestille sig at frøken Jensen mangler en dugholder til havefesten hun skal holde den kommende dag. Hun går ind på nettet, finder en dugholder hun godt kan lide og trykker derefter på print. Afhængig af printeren og kvaliteten af printet, vil dugholderen være klar til brug under havefesten. Dette lyder måske rimelig sci-fi, men er faktisk allerede muligt. I et kommende afsnit beskriver vi de nuværende 3D-printer services.
% % section en_del_af_husstanden (end)

% % chapter omfanget_af_3d_printer (end)