Når man snakker om filtyper og dataformater, i forbindelse med 3D-printning, er der 2 væsentlige typer, som man gør brug af. Den første er selve modellen, som typisk vil være i form af en STL fil eller lignende. 3D-printere kan ikke fortolke modeller direkte, men skal i stedet modtage instrukser, som genereres på en computer. Printerne kan modtage instrukserne fra computeren, ved brug af en direkte datalinje(f.eks USB). Derudover er det også muligt at indlæse instrukserne, fra et lokalt medie på printeren(f.eks SD-kort).
Til at generere instrukserne bruges en slicer, som skriver dem som G-koder. G-koder angiver hastigheden, mængden af plastik og hvilke koordinater det skal følge.