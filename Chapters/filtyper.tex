Når man snakker om filtyper og dataformater i forbindelse med 3D-printning, er der 2 væsentlige typer som man gør brug af. Den første er selve modellen som typisk vil være i form af en STL fil eller lignende. Men printeren kan ikke printe ud fra en model, Den skal i stedet modtage instrukser. Instrukserne genreres på en computer, ud fra modellen. Printerne kan modtage instrukserne fra computeren, ved brug af en direkte datalinje. (fx. USB) Der udover er det også muligt at indlæse instrukserne, fra et lokalt medie på printeren. (fx. SD-kort)
Til at genere instrukserne bruges en slicer, som oftest bruger G-koder som instrukser. G-koder er koordinat sæt som printeren skal kør hen til, og ved hvilken hastighed den skal printe med.