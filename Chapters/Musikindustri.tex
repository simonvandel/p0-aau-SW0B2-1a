\chapter{Musikindustrien}

\section{Digitalisering af en etableret industri}

Hvis man omtaler problemer ved digitaliseringen af markeder, er det umuligt at undgå musikindustrien. Lige siden kampagnen ”Home taping is killing Music” \autocite{andrew_dickson_home_2005} har musikindustrien kæmpet en brav kamp imod kopiering af sange. På trods af, at det gang på gang har vist sig, at nye medier ikke dræber musikken, har musikindustrien forgæves forsøgt at bekæmpe de nye platforme for kopiering, indtil de til sidst tilpasser sig. Det eksempel som, historisk set, virker mest komisk, er muligvis ideen om kassettebåndet der ville udrydde musik. Dog er dette tilfælde også det, som minder mest om det nyeste eksempel, nemlig internettet.

\subsection{Fildeling}

Internet fildeling startede for alvor i 1999, med P2P \footnote{Peer-to-peer} fildelingstjenesten Napster. Først 5 år efter, i 2004, fik industrien endelig fat i online salg af musik, med Itunes Music Store fra Apple Inc. (senere blot Itunes Store). Itunes Music Store blev lanceret i 2003, og blev hurtigt meget populært i løbet af 2004. Udbredelsen af Ipod’en, havde selvfølgelig en stor betydning, men der var stadig problemer andre steder, som f.eks. kvalitet og ejerskab. I takt med at internet hastighederne blev bedre, blev muligheden for kvalitet også, og i 2009 blev det annonceret at Apple ville fjerne store dele af deres DRM \footnote{Digital Rights Management} program. Dette var bl.a. med til at gøre Itunes Store til den mest sælgende udbyder af musik i verden i 2010. \autocite{apple_itunes_2010} Der skulle altså gå næsten et årti, før den etablerede industri tilpassede sig til teknologien. 

\subsection{3D-printerens æra}

Flere eksperter har udtalt at 3D-printerer bliver større end internettet \autocite{financial_times_3d_2012}, og selvom dette er en vovet udtalelse, er der nok noget om sagen. Spørgsmålet i denne sammenhæng er, hvorvidt industrier som bliver påvirket af 3D-printerer, enten som en forretningsmulighed eller som en konkurrent, vil kunne tilpasse sig hurtigere. Vil de gennemgå 10 år med retssager og ransagninger af private hjem, fordi der er blevet kopieret en servietholder, før de formår at inkorporere 3D-printere i deres forretningsmodel.
Den store forskel på fildeling af musik, film og software, og så fildeling af 3D-schematics, er mængden af industrier som potentielt vil blive påvirket. Hvis ikke de påvirkede industrier lærer af tidligere industriers fejl, så kan de komme til at koste dem dyrt. Det er aldrig fordelagtigt for en industri, at bekæmpe deres kundegrupper, en lektie musikindustrien kan skrive under på. Men selv hvis en virksomhed vælger at se igennem fingre med piratkopiering af deres produkter, kan de stå overfor en voldsom forværring af deres konkurrence evne.
Den nuværende økonomi er i forvejen koncentreret meget nært omkring de logistiske aspekter, og med 3D-printere i private hjem vil der for alvor ske en ændring i måden vi tænker logistik på.
