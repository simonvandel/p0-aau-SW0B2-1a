\lstset{ %
language=C++,                % choose the language of the code
basicstyle=\footnotesize,       % the size of the fonts that are used for the code
numbers=left,                   % where to put the line-numbers
numberstyle=\footnotesize,      % the size of the fonts that are used for the line-numbers
stepnumber=1,                   % the step between two line-numbers. If it is 1 each line will be numbered
numbersep=5pt,                  % how far the line-numbers are from the code
backgroundcolor=\color{white},  % choose the background color. You must add \usepackage{color}
showspaces=false,               % show spaces adding particular underscores
showstringspaces=false,         % underline spaces within strings
showtabs=false,                 % show tabs within strings adding particular underscores
frame=single,           % adds a frame around the code
tabsize=2,          % sets default tabsize to 2 spaces
captionpos=b,           % sets the caption-position to bottom
breaklines=true,        % sets automatic line breaking
breakatwhitespace=false,    % sets if automatic breaks should only happen at whitespace
escapeinside={\%*}{*)}          % if you want to add a comment within your code
}

\textbf{STL}
Inden for 3D-modeller findes der utallige filtyper, næsten alle CAD programmer har deres egen. Men skulle man tale om en standart for 3D-modeller, så er STL Den mest universelle filtype. Det er også den mest almindelige indenfor distribuering af modeller til 3D-printning. \cite{makerbot}
Stl filtypen bruges til at gemme informationer om et fysisk 3D-objekter. Stl indeholder kun oplysninger om modellens ydre opbygning, og kan derfor ikke indeholde informationer så som modellens farver. 
Stl filer kan være skrevet i ASCII format. 
En sådan fil er bygget op af tredimensionelle trekanter. 
Hver trekant er beskrevet ved brug af 3 koordinatsæt som er trekantens hjørner, og en retnings vektor til at beskrive hvordan trekanten vender. 
Som nævnt tidligere beskriver en stl kun modellens omrids, og derfor gør er retnings vektoren af vigtig betydning, og beskriver hvordan trekanten vender. \cite{stl}
Hver trekant er beskrives ved brug af normalvektoren og 3 punkter, (X,Y,Z) som udgør trekantens hjørnepunkter. Normalvektoren består også af 3 koordinater (X,Y,Z). Alle punkter skal skrives som Floating points\cite{stl}.
En trekan kaldes for en facet, og skal opstille på følgende måde:
\begin{lstlisting}
          facet normal Nx Ny Nz
            outer loop
              vertex X1 Y1 Z1
              vertex X2 X2 Z2
              vertex X3 Y3 Z3
            endloop
          endfacet
\end{lstlisting}
Facet normal betyder begynd trekant med normal vektor med koordinaterne Nx, Ny, Nz. Outer loop starter indramning af de 3 koordinater til trekantens hjørnepunkter. Hver hjørnepunkt beskrives på følgende måde vertex X Y Z. Punkterne afsluttes med endfacet, efterfulgt af endloop for at afslutte trekanten. 
For at definerer hvor filen starter og slutter bruges syntaksen solid "navn" til start af 3D-model, og endsolid til afslutning af 3D-modellen.
\newpage
Eksempel på en 3D-model. (Simpel firkant) \cite{Stl_Eksempel}
\begin{lstlisting}
        solid cube_corner
          facet normal 0.0 -1.0 0.0
            outer loop
              vertex 0.0 0.0 0.0
              vertex 1.0 0.0 0.0
              vertex 0.0 0.0 1.0
            endloop
          endfacet
          facet normal 0.0 0.0 -1.0
            outer loop
              vertex 0.0 0.0 0.0
              vertex 0.0 1.0 0.0
              vertex 1.0 0.0 0.0
            endloop
          endfacet
          facet normal 0.0 0.0 -1.0
            outer loop
              vertex 0.0 0.0 0.0
              vertex 0.0 0.0 1.0
              vertex 0.0 1.0 0.0
            endloop
          endfacet
          facet normal 0.577 0.577 0.577
            outer loop
              vertex 1.0 0.0 0.0
              vertex 0.0 1.0 0.0
              vertex 0.0 0.0 1.0
            endloop
          endfacet
        endsolid
\end{lstlisting}
De to eneste store problemer ved stl filtypen, er at der igen mulighed for rettigheds beskyttelse er. Samt at der ikke er nogen mulighed for at tilføje farver til modellen.



%    \bibliographystyle{plainnat}
%    \bibliography{Kilder}
