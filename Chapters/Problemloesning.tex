\chapter{Problemløsning} % (fold)
\label{cha:probleml_sning}

\begin{itemize}
	\item Copyright/jura/piratkopiering
	\item Mangel på platforme til egenprintning
	\item Brugervenlighed
	\item Tilgængelighed
	\item Husk: skriv 3D-schematics i stedet for modeller og andet
	\item Husk: sig til alexander: beskriv hvordan man nemt kan dele STL-filer med sine venner, eller andre online
	\item Husk: er det ok at printe samme genstand flere gang, uden at betale flere gange?
\end{itemize}

På baggrund af de omtalte problemer ved de nuværende platforme, har vi tænkt på en mulig løsning. Vi har forsøgt at minimere muligheden for at kunne piratkopiere 3D-schematics. Derudover har vi forsøgt at gøre det nemt og tilgængeligt for ikke-tekniske forbrugere af 3D-printere. Vi målretter derfor vores problemløsning til en bredere målgruppe. Vi kalder idéen for \textbf{CloudSlicing}. 

\section{CloudSlicing} % (fold)
\label{sec:cloudslicing}

\subsubsection{Servicen} % (fold)
\label{ssub:servicen}

CloudSlicing er en online service der tilbyder en designer-til-forbruger platform der muliggøre salg af 3D-schematics. Servicen er designet til at gøre det svært for (piraten), men nemt for brugeren af servicen.

Det vil være muligt at tilgå servicen som en app og en hjemmeside.

Features:

\begin{itemize}
	\item Søgefunktioner
	\item Tilpasning af 3D-schematic (farver, kvalitet og størrelse)
	\item Tidligere køb
	\item Send-til-printer knap
\end{itemize}

% subsubsection servicen (end)



\subsection{Nødvendige specifikationer} % (fold)
\label{sub:krav_til_3d_teknologien}

For at mindske kæden af led i 3D-printning, skal 3D-printeren kunne kommunikere direkte til vores online service. Dette kræver at printeren har konstant adgang til internettet.

Printeren skal have en nem opsætning uden brug af en ekstern computer.

% subsection krav_til_3d_teknologien (end)

\subsection{Hvad løser det?} % (fold)
\label{sub:hvad_l_ser_det_}

Som diskuteret tidligere i denne rapport, skal man bruge et 3D-schematic for at printe et objekt. I stedet for at 3D-schematics bliver slicet lokalt på brugerens computer, slicer vores service denne 3D-schematic i skyen, for derefter at sende G-koden direkte til 3D-printeren. 

Ved at sende G-koden direkte til 3D-printeren, er det ikke en nødvendighed for brugeren at slice 3D-schematics selv. Dette gør muligt at introducere en ``plug-and-print'' funktion, hvilket betyder at brugeren bare skal tænde printeren, etablere internetforbindelse og logge ind med en oprettet CloudSlicing-konto på printeren. 3D-printeren er nu tilsluttet vores service, og er nu klar til at modtage G-koder.

Ved vores løsning modtager brugeren aldrig STL-filerne, men kun de printer-specifikke G-koder. Det vil stadig være muligt at dele G-koderne, men disse G-koder kan kun benyttes af de samme printere som ``piraten'' har.

% subsection hvad_l_ser_det_ (end)



% section cloudslicing (end)

% chapter probleml_sning (end)