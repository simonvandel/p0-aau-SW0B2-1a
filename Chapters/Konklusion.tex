	\chapter{Konklusion}
Grunden til at 3D-print ikke er mere udbredt skyldes primært 3 hovedårsager. Før 3D-print kan nå en bredere målgruppe, er der behov for forbedringer indenfor brugervenligheden. Et af de tekniske problemer er begrænsninger ved nuværende datatyper. Datatyperne har ingen mulighed for rettighedsbeskyttelse hvilket skaber komplikationer ved fremtidig markedsvækst. 
Dertil kommer problemer som grunder i juridisk uklarhed, da denne markedsform er forholdsvis unik. Dette betyder en mangel på retspraksis, og medfører krav om lovjusteringer for bedre at tilse 3D-printermarkedet.
3D-printeres kendskabsgrad er også problematisk lav, hvilket gør det mindre attraktivt for investorer at engagere sig i markedet. Der findes også en prisproblematik hvilket skyldes patenteret nøgledele i printere og høje produktionsomkostninger. 
En mulig løsning til nogen af problemerne, er en idé kaldet "Cloudslicing", som er en platform der øger brugervenligheden og simplificere juridiske aspekter.