\chapter{Juridiske Problemstillinger}

\section{Juridiske Problemstillinger}

Et af de største spørgsmål forbundet med 3D-printerens udvikling, er de juridiske komplikationer. Man har tidligere set store konflikter ved digitalisering af intellektuel ejendom, især indenfor film og medie industrien. Hvorvidt 3D-printeren vil medføre konflikter af samme proportion er svært at sige, men det er sikkert at der vil opstå usikkerhed omkring lovgivningen på området.

\subsection{Gældende Lovgivning}

Ifølge designlovens § 10, stk. 1 fremgår det tydeligt, at designret ikke kan udøves ved; ”handlinger, der foretages i private øjemed”\cite{jura1}, og ophavsretslovens § 12 giver lov til at fremstille enkelte eksemplarer til private brug. Dog påpeges det i § 12 stk. 4 af ophavsretsloven\cite{jura2}, at § 12 stk. 1 ikke giver ret til kopiering til privat forbrug af bl.a. brugskunst, hvis der indgår benyttelse af fremmed medhjælp ved eksemplarfremstillingen.
Hvis en mulig fremtid bliver at have printerstationer rundt omkring i landet\cite{jura3}, som forbrugere kan tilgå, vil det kræve en overvågning af printet materiale, hvis denne lov skal kunne håndhæves. Det er ikke usandsynligt at en højesterets dom, vil kunne kræve en slags overvågning af hvad der bliver printet, selvom det ikke er printerstationens ejer, der foretager den ulovlige handling. Her tænkes der især på sagen ’Telenor mod IFPI’ (med flere)\cite{jura4}, hvor Telenor blev pålagt at hindre adgang til Thepiratebay.org.

\subsection{Retspraksis}

Der findes pt. ingen højesteretsdomme som omtaler kopi af design i private andre ikke-kommercielle hensigter\cite{jura5}. Dette er forståeligt eftersom der ikke findes et stort behov for retspraksis indenfor dette emne, men med 3D-printere under voldsom udvikling, er det næsten kun et spørgsmål om tid. Det er tydeligt at den første sag om 3D-print, kommer til at have store konsekvenser for følgende sager, i samme stil som Telenor mod IFPI dommen, fik det for andre danske internetudbydere\cite{jura6}.
Edb-programmer er sidestillet med litterære værker når det gælder ophavsretsloven\cite{jura7}, så det er nærtliggende at betragte STL filer, og andre printerfiler, som litterære værk, med dertil tilhørende ophavsret. Men eftersom en STL fil, blot er en slags opskrift, er det derfor kun opskriften og ikke selve produktet som er beskyttet. Dette kan medføre et senarie, hvor en skanning af et ophavsretsbeskyttet design bliver lagt ud til fri benyttelse, hvorefter det vil være lovligt at 3D-printe fra denne skanning til privat forbrug.

\subsection{Lovgivningens utilstrækkelighed}
 
Dette eksempel viser at den nuværende lovgivning ikke er tilstrækkeligt teknologisk fokuseret, til at indfatte et boom indenfor 3D-printere. Det er ikke utænkeligt at en kontroversiel højesteretsdom (eller sågar en byretsdom), vil medføre et endnu større pres på Christiansborg fra de danske domstole, og efterspørgsel efter juridisk klarhed omkring digitale medier og nyere teknologi.
Det er i hvert fald tydeligt, at den nuværende lovgivning ikke er tilstrækkeligt dækkende, og indtil dette ændrer sig, vil det være meget vanskeligt at forudse 3D-printningsmarkedet.
