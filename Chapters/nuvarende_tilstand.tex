\chapter{Nuværende tilstand af 3D-print} % (fold)
\label{cha:nuv_rende_tilstand_af_3d_print}


\subsubsection{Hvordan virker 3D-printere teknisk?}

En 3D-printer er en printer som kan lave solide objekter, skabt fra en 3D-model. En 3D-prniter laver tredimensionelle computer modeller om til  fysiske objekter skabt i plastik, gips eller metal. Langt de fleste printer printer i solid plastik som abs, en plastik type som man kender fra freksembel Lego klodser. Der findes forskellige måder at printe på, men den mest almindelige er additive manufacturing også kaldet lag på lag metoden. additive manufacturing betyder at noget byges lag på lag, printerens fortolkning af additive manufacturing betyder at printeren læger plastik lag på plastik og bygger derved sin objekter fra bunde og op. 

\subsubsection{Printer typer}

Plastik printer er nok den mest almindelige printer, og sandsynlig vis også den som man i nær fremtid ville kunne opleve, i den almindelige danskers hjem. Der er findes mange typer plastik printer, og mange forskellige typer plastik som kan bruges til 3d-print. De to førende plastik typer til hobby printer og halv professionelle 3D-Printer er PLA og ABS. De to plastik typer kan købes i ruller klar til at printe med, og så er de så billige at selv den almindelige dansker ville kunne være med. 
Denne type printer funger ved at en dyse varmes op til plastikens smelte temperatur, mellem 180-210, for begge plastik typer, Derefter presse en tynd snor af plastik ud gennem dysen, mens dysen flytter sig i 2 akser for at tegne det første lag, med den smeltede plastik. Når dysen har lagt første plastik lag, så kører planet som plastikken ligger på ned, for at dysen i gen kan begynd at lægge et nyt plastik lag, denne process gentages ind til printet er færdigt. Afstand som planet kører ned med bestemmer kvaliteten af printet, i takt med at afstand som planet kører ned forstøres, forværres kvaliteten af printet.  Hvor imod at en mindre afstand mellem hver lag, giver højere kvalitet og længer print tid. Printer har oftest en blæser til at køle spidsen af dysen, så den flydende plastik størkner hurtigst muligt efter at have forladt dysen. Plastikens størkne hastighed er også afgørende for hvor hurtigt en printer kan kører. 

Metal printere funger på en lidt anden vis, de bruger for det første ikke en tråd til at printe med, som så mange plastik printer. Metal printer bruger metal pulver, som bliver limet sammen lag på, sammen princip som plastik printerne. 
Mens printer limer metal pulveret sammen, lægger den sand de steder, hvor der ikke skal printes metal. Sandet bruges som support til det rigtige metal. Det vil sige at har printeren eksempelvis bruge for at printe steder, i luften hvor der ikke er metal pulver under, så forhindre suport sand metallet i at falde ned, og lande steder hvor det ikke var tiltænk at der skulle være metal.
Når den er færdig med at printe skal printet bages, og derefter kan support sandet typisk fjernes med vand, eller luft tryk, og en print er færidgt.

\subsubsection{Mærkevare}

Når man har et produkt som ligger under et brand kaldes det for en mærkevare. Det behøves ikke nødven-digvis være et tøjmærke da det dækker over et meget stører marked.  En stol fra IKEA eller bil fra Skoda, er begge kategoriseret som en mærkevare.  
Ofte kan et mærke genkendes med et tegn eller ikon, og ifølge varemærkeloven: § 2 "bestå af alle arter tegn, der er egnet til at adskille en virksomheds varer eller tjenesteydelser fra andre virksomheders, og som kan gengives grafisk". På den måde kan man genkende kvalitet eller prisniveau fra et produkt.
Kan denne samme form for identitet opnås ved en 3D-printer, når produktet bliver lavet af forbrugeren fra sit hjem?

Man kan sige at hvis producenten selv laver designet som printeren skal følge, må produktet som printeren printede være en del mærket. Eksempelvis: IKEA sælger et schematic til en dugholder. Det bliver købt af en forbruger der med det samme sætter sin printer til at lave en kopi. Ved at forbrugeren nu bruger IKEA’s opskrift på at lave det produkt, kunne man argumentere for at det er under samme varemærke.
Man kunne også argumentere for at det ikke hænger sådan sammen. Hvis NOVA designer en ny sweater til deres efterårskollektion og bestemmer sig for at udgive deres opskrift, og lader derved andre lave deres NOVA sweater. Hvis en privat person så bestemmer sig for at strikke denne sweater, og følger opskriften vil den stadigvæk ikke anses som at være en del af mærket. Selvom at sweateren er lavet ud fra samme opskrift som den virksomheden bruger, er den ikke lavet med samme standarder som kan findes på virksomheden. 
Men man kan sige at det ikke gælder for 3D-printning da det bliver lavet mekanisk og er derfor ikke påvirket af brugens input. Når produktet bliver printet kan det være 100\% identisk til de varer der sælges i butikkerne, hvis der da bruges samme farver og materialer som begge steder. 
Nogle brugere vil dog kun anderkende et produkt som at være en del af branded hvis det reelt kommer fra producenten der ejer mærket. Det ser man når folk køber højkvalitets kopivarer. Det er ikke helt det samme da dem der kopirer ofte har lavet deres design så det har skulle ligne, hvor ved 3D-prinitng kan aves ud fra producentens egen opskrift eller design.



% chapter nuv_rende_tilstand_af_3d_print (end)
