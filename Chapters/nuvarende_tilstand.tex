\chapter{Nuværende tilstand af 3D-print} % (fold)
\label{cha:nuv_rende_tilstand_af_3d_print}


\subsubsection{Hvordan virker 3D-printere teknisk?}
			
En 3D-printer laver tredimensionelle computer modeller om til fysiske objekter skabt i plastik, gips eller metal. Langt de fleste 3D-printerer, printer i solid plastik som ABS, en plastik type som man kender fra bl.a. Lego klodser. Der findes forskellige måder at printe på, men den mest almindelige er Additive Manufacturing, også kaldet lag på lag metoden. Additive Manufacturing betyder at der printes plastik lag på plastik lag, og derved bygger objekter fra bunde og op. 

\subsubsection{Printer typer}
FDM (Fused deposition modeling) printere er den mest almindelige printer til privat brug. Der er findes mange typer FDM-printerer, og mange forskellige typer plastik som bruges på disse printerer. De to førende plastiktyper til hobbyprinterer og semiprofessionelle 3D-Printerer er PLA og ABS. 

%hvad består en 3d printer af? hvilke dele ingår etc. (måske en blueprint tegning?)

FDM-printere fungerer ved at dysen varmes op til plastikens smeltepunkt, mellem 180 og 210 grader, for begge plastik typer. Derefter presses en tynd tråd af plastik ud gennem dysen, mens dysen flytter sig i 2 akser for at tegne det første lag, med den smeltede plastik. 
Når dysen har lagt det første lag plastik, sænkes planen som der printes på, for at dysen igen kan begynd at printe et nyt lag plastik. Denne process gentages ind til printet er færdigt. 
Afstand som planen sænkes med bestemmer kvaliteten af printet, i takt med at afstand forøges falder klvaliteten af printet og printtiden formindskes.

3D-Printerer har oftest en blæser til at køle spidsen af dysen, så den flydende plastik størkner hurtigst muligt efter at have forladt dysen. Plastikens størknehastighed(er det et ord?) er også afgørende for hvor hurtigt en printer kan printe. 

DMLS-printerer (Direct metal laser sintering) fungerer på en anden metode end FDM-printere. DMSL-printerer bruger en laser til at bind metalpulver sammen. Et lag af metal pulver lægges ud på print planet, eller oven på et af de tideligere printet lag. Derefter bruges en laser til at en varmer et koncentreret område af printet op, og smelter dette punkt fast til laget under. På steder hvor der ikke skal printes metal, udfyldes der sand istedet til at understøtte, eventuel lag af metal som kan forkomme over dette punkt.
Når den er færdig med at printe skal printet bages(teknisk term?), og derefter kan støttesandet typisk fjernes med vand eller lufttryk. Printet derefter er færdigt.

%mangler en  afrundning. evt. hold dig til plastik printerer og måske kun de metoder som er fælles for printerer som kan ende i et hjem. FOKUS PÅ HJEMME PRINT! :D

%medtag evt. metode om SLA

\subsubsection{Mærkevare}

Når man har et produkt som ligger under et brand kaldes det for en mærkevare. Det behøves ikke nødvendigvis være et tøjmærke, da det dækker over et meget stører marked.  En stol fra IKEA eller bil fra Skoda, er begge kategoriseret som en mærkevare.  
Ofte kan et mærke genkendes med et tegn eller ikon, og ifølge varemærkeloven: § 2 "bestå af alle arter tegn, der er egnet til at adskille en virksomheds varer eller tjenesteydelser fra andre virksomheders, og som kan gengives grafisk" . På den måde kan man genkende kvalitet eller prisniveau fra et produkt.
Kan denne samme form for identitet opnås ved en 3D-printer, når produktet bliver lavet af forbrugeren fra sit hjem?

Man kan sige at hvis producenten selv laver designet som printeren skal følge, må produktet som printeren printede, være en del af mærket. Eksempelvis: IKEA sælger et schematic til en dugholder. Det bliver købt af en forbruger der med det samme sætter sin printer til at lave en kopi. Ved at forbrugeren nu bruger IKEA’s opskrift på at lave det produkt, kunne man argumentere for at det er under samme varemærke. NOVA designer en ny sweater til deres efterårskollektion og bestemmer sig for at udgive deres opskrift i deres butikker. De gøres ved at man køber en pose med det garn der skal bruges dertil sammen med en strikkeopskrift. Forbrugeren kan da gå hjem og gå i gang med at strikke denne sweater og til sidst sy det medfølgende mærke og/eller varemærke fast. På den måde går producenten med til at vise at denne sweater er en del af deres brand, selvom at den er lavet af en privat person. 

Men man kan sige at det ikke gælder for 3D-printning da det bliver lavet mekanisk og er derfor ikke påvirket af brugens input. Når produktet bliver printet kan det være 100\% identisk til de varer der sælges i butikkerne, hvis der da bruges samme farver og materialer som begge steder. 
Producenten kunne også vælge at fjerne deres garanti eller fortrydelsesret til et produkt. Hvis de mener at brugeren ikke har brugt et materiale de har anbefalet, eller brugt lav kvalitets printer. 
Nogle brugere vil dog kun anderkende et produkt som at være en del af branded, hvis det reelt kommer fra producenten der ejer mærket. Det ser man når folk køber højkvalitets kopivarer. Det er ikke helt det samme, da dem der kopirer ofte har lavet deres design for at det skulle ligne originalenerne, hvor ved 3D-prinitng kan laves ud fra producentens egen opskrift eller design.


% chapter nuv_rende_tilstand_af_3d_print (end)
