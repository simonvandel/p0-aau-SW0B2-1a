%!TEX root = ../Master.tex
\section{Tidshorisont for 3D-printere}
\subsection{Konsekvenser for andre industrier}


3D-printere har eksisteret i industrien i lang tid, hvor de er blevet brugt til f.eks. at lave hurtige prototyper. \footcite[105]{manyika_disruptive_2013} Nu hvor teknologien er blevet bedre, er 3D-printere blevet hurtigere, mere præcise og også billigere. Dette har gjort det muligt for forbrugere at købe personlige 3D-printere. 


Produkter som legetøj og smykker kan printes der hjemme. Forbrugeren kan endda selv vælge og bestemme egenskaber af det, de printer. Egenskaber som farve, størrelse, form og materiale kan alt sammen bestemmes af brugerens præferencer. Producenter af f.eks. legetøj og smykker kan ikke tilbyde samme grad service. Det kan betyde at disse producenter bliver nødt til at være nytænkende, og finde nye måder hvorpå man kan konkurrere med hjemmeprintning.
s
På samme tid kan producenter også gøre brug at teknologien -{} til at lave komponenter som aldrig før var muligt. Det er nu muligt at printe komponenter inde i komponenter, hvilket skaber spændende muligheder.
I takt med at man kan printe i flere og flere forskellige materialer som gummi, sukkermasse, kød og metaller, er der også flere virksomheder som investerer i 3D-printning.
Design Reality laver gasmasker til militært brug og brandfolk. Dette skulle forbedre deres standarder, grundet en ny amerikansk lovgivning omkring sikkerheds standarder. De valgte at benytte 3D-print til at løse denne opgave, da de på intet mindre en 5 hverdage kunne designe et helt nyt produkt, til at blive printet og testet på den anden side af kloden. \autocite{3ders.org_design_2013}

\subsection{Mærkevare}

Når man har et produkt der adskiller sig fra andre tilsvarende produkter på markedet, kan det kaldes for en mærkevare. Typisk er disse produkter også tilknyttet et kendt brand.
Det behøves ikke nødvendigvis være et tøjmærke, en stol fra IKEA eller bil fra Skoda, er begge kategoriseret som en mærkevare. 

Ofte kan et mærke genkendes med et tegn eller ikon, og ifølge varemærkeloven: § 2 "bestå af alle arter tegn, der er egnet til at adskille en virksomheds varer eller tjenesteydelser fra andre virksomheders, og som kan gengives grafisk". På den måde kan man genkende kvalitet eller prisniveau af et produkt. 
Kan man opnå den samme form for mærkeidentitet, når produktet bliver printet af en 3D-printer hjemmefra?

Man kan sige at hvis producenten selv laver designet som printeren skal følge, må det printet produkt, være en del af mærket. Eksempelvis: IKEA sælger et schematic til en dugholder. Det bliver købt af en forbruger, som med sin egen printer laver en fysisk kopi. Ved at forbrugeren bruger IKEA’s opskrift til at lave kopien, kunne man argumentere for at det er under samme varemærke. 

Hanne Falkenbergs\autocite{hanne_falkenberg_hanne_????} designer en ny sweater til deres efterårskollektion og bestemmer sig for at udgive deres opskrift i deres butikker. Forbrugeren køber en pose med det garn der skal bruges dertil sammen med en strikkeopskrift. Forbrugeren kan da gå hjem og gå i gang med at strikke sweater efter opskriften, sidst påføres det medfølgende logo på sweateren. Forbrugeren får det materialer med som producenten selv vil bruge under produktionen. Mærkets materialekvalitet vil ofte blive set som at være en del af branded.  På den måde giver producenten udtryk for at produktet er en del af deres brand, selvom at den er produceret af en privat person. 
																
Gælder de samme præncipper for 3D-printning da det produceres mekanisk og er derfor ikke påvirket af brugens input. Når produkter bliver printet kan de være 100\% identiske til de varer der sælges i butikkerne, hvis der gøres bruge af de samme kvalitet, farver og materialer.
Producenter kunne vælge at fjerne deres garanti eller fortrydelsesret til et produkt. Hvis de mener at brugeren ikke har brugt et materiale de har anbefalet, eller brugt lav kvalitets printer.
Nogle brugere vil dog kun anderkende et produkt som at være en del af branded, hvis det reelt kommer fra producenten der ejer mærket. Det ser man blandt andet når man køber højkvalitets kopivarer.\autocite{_vi_????}
Det er ikke helt det samme, da dem der kopierer ofte har lavet deres design for at det skulle ligne originalenerne, hvor 3D-print kan laves ud fra producentens originale opskrift eller design.