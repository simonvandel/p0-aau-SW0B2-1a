
\chapter{Tidshorisont for 3D-printere}
\section{Konsekvenser for andre industrier}


Produkter som legetøj og smykker, kan printes der hjemme. Forbrugeren kan endda selv vælge og bestemme egenskaber af det de printer. Egenskaber som farve, størrelse, form og materiale kan alt sammen bestemmes af brugerens præferencer. Producenter af f.eks. legetøj og smykker kan ikke tilbyde sammen grad service. Det kan betyde at disse producenter bliver ny til at være nytænkende og finde nye måde hvorpå at man kan konkurrere med hjemmeprintning.

På samme tid kan producenter også gøre brug at teknologien, til at lave komponenter som aldrig før var muligt. Det er nu muligt at kunne printer komponenter inde i komponenter uden brug af samling, som skaber spændende muligheder.
I takt med at man kan printe i flere og flere forskellige materialer som gummi, sukkermasse, kød og metaller. Er der også flere virksomheder som investerer i 3D-printning.
Design Reality laver gasmasker til militær brug og brandfolk, skulle forbedre deres standarder grundet en ny amerikansk lovgivning omkring sikkerheds standarder. De valgte at benytte 3D-print til at løse den opgave, på intet mindre en 5 hverdage havde de design et nyt klar produkt til at blive printet og testet på den anden side af kloden.\cite{gasmasker}



\section{Behov for den almene slutbruger}

Billigt
Nemt
Det reele behov


Maslows behovspyramide \cite{abraham_harold_maslow_theory_????} beskriver et individs behov. For at 3D-printning skal blive en succes i det almene hjem, kan 3D-printning f.eks. introduceres på samme måde som iPad'en blev det af Apple. Apple skabte et behov for forbrugeren ved at appellere til nogle behov i behovspyramiden. Grundessensen i iPad præsentationen var at iPad'en kan udføre nogle eksisterende opgaver, bedre end de eksisterende produkter allerede kan.

Ligesom iPad'en, er 3D-printeren et medium til at opfylde behov via andre produkter. 