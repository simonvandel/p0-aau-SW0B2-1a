%!TEX root = ../Master.tex
\section{Tidshorisont for 3D-printere}
\subsection{Konsekvenser for andre industrier}


3D-printere har eksisteret i industrien i lang tid, hvor de er blevet brugt til f.eks. at lave hurtige prototyper. \footcite[105]{manyika_disruptive_2013} Nu hvor teknologien er blevet bedre, er 3D-printere blevet hurtigere, mere præcise og også billigere. Dette har gjort det muligt for forbrugere at købe personlige 3D-printere. 


Produkter som legetøj og smykker kan printes der hjemme. Forbrugeren kan endda selv vælge og bestemme egenskaber af det, de printer. Egenskaber som farve, størrelse, form og materiale kan alt sammen bestemmes af brugerens præferencer. Producenter af f.eks. legetøj og smykker kan ikke tilbyde samme grad service. Det kan betyde at disse producenter bliver nødt til at være nytænkende, og finde nye måder hvorpå man kan konkurrere med hjemmeprintning.
s
På samme tid kan producenter også gøre brug at teknologien -{} til at lave komponenter som aldrig før var muligt. Det er nu muligt at printe komponenter inde i komponenter, hvilket skaber spændende muligheder.
I takt med at man kan printe i flere og flere forskellige materialer som gummi, sukkermasse, kød og metaller, er der også flere virksomheder som investerer i 3D-printning.
Design Reality laver gasmasker til militært brug og brandfolk. Dette skulle forbedre deres standarder, grundet en ny amerikansk lovgivning omkring sikkerheds standarder. De valgte at benytte 3D-print til at løse denne opgave, da de på intet mindre en 5 hverdage kunne designe et helt nyt produkt, til at blive printet og testet på den anden side af kloden. \autocite{3ders.org_design_2013}



