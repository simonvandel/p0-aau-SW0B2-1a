
\chapter{Tidshorisont for 3D-printere}
\section{Vil det have nogle konsekvenser for andre industrier.}


Produkter som legetøj og smykker kan printes der hjemme. Forbrugeren kan endda selv vælge og bestemme egenskaber af det, de printer. Egenskaber som farve, størrelse, form og materiale kan alt sammen bestemmes af brugerens præferencer. Producenter af f.eks. legetøj og smykker kan ikke tilbyde samme grad service. Det kan betyde at disse producenter bliver nødt til at være nytænkende, og finde nye måder hvorpå man kan konkurrere med hjemmeprintning.

På samme tid kan producenter også gøre brug at teknologien -{} til at lave komponenter som aldrig før var muligt. Det er nu muligt at printe komponenter inde i komponenter, hvilket skaber spændende muligheder.
I takt med at man kan printe i flere og flere forskellige materialer som gummi, sukkermasse, kød og metaller, er der også flere virksomheder som investerer i 3D-printning.
Design Reality laver gasmasker til militært brug og brandfolk. Dette skulle forbedre deres standarder, grundet en ny amerikansk lovgivning omkring sikkerheds standarder. De valgte at benytte 3D-print til at løse denne opgave, da de på intet mindre en 5 hverdage kunne designe et helt nyt produkt, til at blive printet og testet på den anden side af kloden. \cite{3ders.org_design_2013}

\section{Hvornår forventes 3D-printere at blive udbredt blandt almene slutbrugere}

Efterhånden som diverse patenter på 3d printer teknikker bliver forældet, derfor vurderer \cite{manyika_disruptive_2013} at vi vil se et enormt boom i 3d printer verdenen. Det vil medbringe mere konkurrence til branchen, som betyder mere overkomlige priser for personlige 3d printere, flere virksomheder vil gøre brug af 3d printer teknologien, da den har den kæmpe fordel at du i stor stil sparer wasteprodukt. 

\section{Hvilke krav har den almene slutbruger}

Staples har en service i kan du indsende dine egne designs til en 3d-printerstation, den printer din model, som du så kan hente eller få sendt hjem. Dennne service er dog ikke særlig attraktiv da den er rimelig dyr, sammenlignet med en tilsvarende masseproduceret vare du kan købe i supermarkedet, uden at skulle vente på at den printes, da denne process godt kan være tidskrævende.
Hjemme 3d-printerer til forbrugeren, har også sine mangler for at kunne blive en succes. Det er et spørgsmål om kvalitet, brugervenlighed og et behov for at eje en 3d printer. Arkitekter og ingeniører kan et behov for at skulle printe en bygning, en konstruktion eller andet. Hvor den gennemsnitlige menneske, der måske ikke lige arbejder med 3d figurer eller modeller, som sådan ikke har brug for at kunne lave prototyper, ikke har et behov for en dyr 3d printer. Uden erfaring med 3d-printere kan det være svært for forbrugeren at kunne gøre brug af printeren. \cite{_3d_2013}
