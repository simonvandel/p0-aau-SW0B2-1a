
\chapter{Tidshorisont for 3D-printere}
\section{Konsekvenser for andre industrier}


Produkter som legetøj og smykker kan printes der hjemme. Forbrugeren kan endda selv vælge og bestemme egenskaber af det, de printer. Egenskaber som farve, størrelse, form og materiale kan alt sammen bestemmes af brugerens præferencer. Producenter af f.eks. legetøj og smykker kan ikke tilbyde samme grad service. Det kan betyde at disse producenter bliver nødt til at være nytænkende, og finde nye måder hvorpå man kan konkurrere med hjemmeprintning.

På samme tid kan producenter også gøre brug at teknologien -{} til at lave komponenter som aldrig før var muligt. Det er nu muligt at printe komponenter inde i komponenter, hvilket skaber spændende muligheder.
I takt med at man kan printe i flere og flere forskellige materialer som gummi, sukkermasse, kød og metaller, er der også flere virksomheder som investerer i 3D-printning.
Design Reality laver gasmasker til militært brug og brandfolk. Dette skulle forbedre deres standarder, grundet en ny amerikansk lovgivning omkring sikkerheds standarder. De valgte at benytte 3D-print til at løse denne opgave, da de på intet mindre en 5 hverdage kunne designe et helt nyt produkt, til at blive printet og testet på den anden side af kloden. \cite{3ders.org_design_2013}





\section{Behov for den almene slutbruger}

Vi har igennem de fleste kapitler, snakket om 3D-printeren som værende det nye internet. Men som med så mange andre opfindelser, er det ikke altid lige nemt for den almene dansker, at se deres behov for produktet. Et klassisk eksempel på dette var Ipad'en, som flere steder blev kaldet en forvokset Iphone, der ikke kunne noget nyt. Nu har den fundet sin vej ind i mange danske hjem, og de fleste bliver brugt i stor stil. 
Vi er ikke i tvivl om hvorvidt 3D-printeren har samme potentiale, hvis end ikke et større et, men hvilke behov opfylder den hos slutbrugeren? En undersøgelse viser, at 80 procent ikke ved hvad en 3D-printer kan, og kun 7 procent af dem, viste interesse efter en kort forklaring.
Men hvordan kan dette hænge sammen med eksperternes udtalelser omkring 'større end internettet'?(evt kilde) Her minder det igen om Ipad'en. Ipad'en var ikke den første tablet på markedet, men hvis du spørger folk, vil de fleste nok opfatte den som den første. Hvorfor? fordi den blev solgt af markedstingsfolk. Det samme behov har 3D-printeren. Den almene slutbruger skal fortælles sit behov for 3D-printeren, og forklares hvor i sin hverdag at den passer ind.

Et stort behov for slutbrugerne er prisen. Vi har ikke foretaget markedsundersøgelser, men hvis folk skal give mere end tusind kroner, vil de gerne have en god idé om produktets muligheder. I skrivende stund er de fleste 3D-printere forholdsvis indviklet, og kræver ofte teknologisk snilde der overskrider det normale behov. Normale printerer er i forvejen for indviklede for mange mennesker, og hvis 3D-printerene skal kunne sælges, skal de gøres meget mere brugervenlige.

Det måske største problem 3D-printeren har, er dens indirekte behovsopfyldelse. Hvis man kigger på Maslow's behovspyramide \cite{abraham_harold_maslow_theory_1943} opfylder printeren direkte, ikke mange behov for den almene slutbruger. Endnu engang trækkes der dog paraleller til Ipad'en, men også til computeren og fjernsynet, der alle er medier for andre produktes evne til at opfylde behov. Dette er et meget vigtigt aspekt, og hvis 3D-printeren nogensinde skal nå ud til en almen slutbruger, så er det af allerhøjeste prioritet, at få taget hånd om dette.