%!TEX root = ../Master.tex
\lstset{ %
language=C++,                % choose the language of the code
basicstyle=\footnotesize,       % the size of the fonts that are used for the code
numbers=left,                   % where to put the line-numbers
numberstyle=\footnotesize,      % the size of the fonts that are used for the line-numbers
stepnumber=1,                   % the step between two line-numbers. If it is 1 each line will be numbered
numbersep=5pt,                  % how far the line-numbers are from the code
backgroundcolor=\color{white},  % choose the background color. You must add \usepackage{color}
showspaces=false,               % show spaces adding particular underscores
showstringspaces=false,         % underline spaces within strings
showtabs=false,                 % show tabs within strings adding particular underscores
frame=single,           % adds a frame around the code
tabsize=2,          % sets default tabsize to 2 spaces
captionpos=b,           % sets the caption-position to bottom
breaklines=true,        % sets automatic line breaking
breakatwhitespace=false,    % sets if automatic breaks should only happen at whitespace
escapeinside={\%*}{*)}          % if you want to add a comment within your code
}

\paragraph{G-koder} % (fold)
\newline
G-koder er et sprog som CNC maskiner(f.eks fræsere) og 3D-printer benytter sig af. Det er instrukser som beskriver en rute, som boret eller printerhovedet bevæger sig i. \cite{gkode} 
Der findes flere hundrede forskellige G-kode funktioner, men til 3D-printning benyttes kun én funktion. Den funktion som printeren bruger hedder G1. G1 fortæller printeren, at den skal bevæge printerhovedet i en lineær strækning fra den nuværende position, til det givne koordinat.
En typisk G-kode kunne se ud som følgende: 

\begin{verbatim}
	G1 X0 Y0 Z20 E0
\end{verbatim}


Denne kode fortæller printeren, at den skal køre til følgende koordinat: "(X, Y, Z) 0,0,20", hvor E-koefficienten er mængden af tråd, der samtidig skal føres ud gennem dysen.
En parameter mere kan tilføjes til G-koden, F-koefficient, som bestemmer med hvilken hastighed ruten skal følges. Er der ikke angivet en hastighed, bruges samme hastighed som den tidligere rute. 
Hver G-kode adskilles ved brug af white spacing.
Et eksempel på G-koder(en simpel firkant):
\begin{lstlisting}
G1 F200 X0 Y0 Z0 E0
G1 F200 X100 Y0 Z0 E100
G1 F200 X100 Y100 Z0 E100
G1 F200 X0 Y100 Z0 E100
G1 F200 0 0 Z0 E100
\end{lstlisting}