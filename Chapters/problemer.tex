\chapter{Problemer} % (fold)
\label{cha:problemer}

\subsection{Vores interesse/krav} % (fold)
\label{sub:vores_interesse_krav}

For at opretholde en tilfredsstillende brugeroplevelse for forbrugeren, opstiller vi nogle grundlæggende krav til 3D-printerne vi understøtter.

Da vores slutbrugere ikke forventes at kende til den tekniske del af 3D-printing, skal 3D-printerens hardware være så simpel som muligt. Dette matcher med vores overordnede mål, om at designe vores slutprodukt med filosofien KISS (Keep It Simple Stupid) i mente.

\subsubsection{Hardware} % (fold)
\label{ssub:hardware}

3D-printerens ydre skal være så simpelt som muligt. Der skal altså kun være de mest basale funktioner tilgængelig for brugeren. Den eneste krævede knap er en ``stop'' knap, så brugeren kan stoppe en igangværende operation.

% subsubsection hardware (end)

\subsubsection{Software} % (fold)
\label{ssub:software}

Brugervenlighed er et vigtigt aspekt i vores løsning. Vi ønsker derfor at 3D-printerens software skal have mulighed for at forbinde sig til vores server. Dette ønskes af 2 grunde:

\begin{itemize}
	\item Ved en direkte kontakt mellem 3D-printer og vores server, undgår vi at brugerne skal installere 3D-printeren på deres PC, da serveren fodrer 3D-printeren direkte.
	\item 3D-printeren kan selv ``melde'' dens printermuligheder til vores server. Dette sikrer en problemfri opsætning af 3D-printer på vores hjemmeside, da slutbrugeren nu ikke længere behøver at kende fabrikanten og modellen af 3D-printeren.
\end{itemize}

% subsubsection software (end)

% subsection vores_interesse_krav (end)

% chapter problemer (end)