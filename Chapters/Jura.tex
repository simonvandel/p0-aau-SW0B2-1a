

\section{Juridiske Problemstillinger}

Et af de største spørgsmål forbundet med 3D-printerens udvikling, er de juridiske komplikationer. Man har tidligere set store konflikter ved digitalisering af intellektuel ejendom, især indenfor film og medie industrien. Hvorvidt 3D-printeren vil medføre konflikter af samme proportion er svært at sige, men det er sikkert at der vil opstå usikkerhed omkring lovgivningen på området.

\subsection{Gældende Lovgivning}

Ifølge designlovens § 10, stk. 1 fremgår det tydeligt, at designret ikke kan udøves ved "handlinger, der foretages i private øjemed" \autocite{retsinformation.dk_designloven_2012}.

Ophavsretslovens § 12 giver lov til at fremstille enkelte eksemplarer til private brug.
Dog påpeges det i § 12 stk. 4 af ophavsretsloven \autocite{retsinformation.dk_ophavsretsloven_2010}, at § 12 stk. 1 ikke giver ret til kopiering til privat forbrug af bl.a. brugskunst, hvis der indgår benyttelse af fremmed medhjælp ved eksemplarfremstillingen.


Hvis en mulig fremtid indeholder, at have printerstationer rundt omkring i landet \autocite{bjorn_godske_dansk_2012} som forbrugere kan tilgå, vil det kræve en overvågning af printet materiale, såfremt printerstationer falder ind under stk. 4's henvisning af fremmed medhjælp. Det er ikke usandsynligt at en højesterets dom, vil kunne kræve en slags overvågning af hvad der bliver printet, selvom det ikke er printerstationens ejer, der foretager den ulovlige handling. Her tænkes der især på sagen "Telenor mod IFPI" (med flere) \autocite{domstol.dk_telenor_2010}, hvor Telenor blev pålagt at hindre adgang til Thepiratebay.org.

\subsection{Retspraksis}

Der findes pt. ingen højesteretsdomme som omtaler kopi af design i private andre ikke kommercielle hensigter \autocite{domstol.dk_domme_????}. Dette er forståeligt eftersom der ikke findes et stort behov for retspraksis indenfor dette emne, men med 3D-printere under voldsom udvikling, er det næsten kun et spørgsmål om tid. Det er tydeligt at den første sag om 3D-print, kommer til at have store konsekvenser for følgende sager, i samme stil som Telenor mod IFPI dommen, fik det for andre danske internetudbydere \autocite{casper_ulsoe_pirate_2010}.
Edb-programmer er sidestillet med litterære værker når det gælder ophavsretsloven \autocite{retsinformation.dk_ophavsretsloven_2010}, så det er nærtliggende at betragte 3D-schematics, og andre printerfiler, som litterære værker, med dertilhørende ophavsret. Men eftersom en 3D-schematic, blot er en slags opskrift, er det derfor kun opskriften og ikke selve produktet som er beskyttet. Dette kan medføre et senarie, hvor en skanning af et ophavsretsbeskyttet design bliver lagt ud til fri benyttelse, hvorefter det vil være lovligt at 3D-printe fra denne skanning til privat forbrug.

\subsection{Våbenloven}

Et andet vigtigt juridisk aspekt, er printning af genstande der er ulovlige i sig selv, det mest åbentlyse eksempel værende våben.

Flere funktionsdygtig våben er blevet fremstillet ved hjælp af en privatejet 3D-printer. To eksempler på våben, som er blevet printet og affyret, er AR-15 (1), en semiautomatisk riffel, og The Liberator(2), en håndpistol. 

Schematic til The Liberator kan downloades fra nettet og printes på de fleste 3D printer. Dette stiller krav til den danske lovgivning, og dens nuværende håndtering af produktion af våben.

I USA har Cody Wilson(3) startet organisationen 'Defence Distributed'(4), som har til formål at dele opensource 3D-schematics til våben. Organisationsen henviser til den amerikanske forfatnings stillingstagen til privates ret til våben(5). Selvom amerikanske myndigheder har forsøgt at fjerne 3D-schematic af The Liberator, er den allerede blevet spredt på internettet, og er tilgenlig for danske borgere.

ifølge dansk våbenlovgivning(6), er det ulovligt at fremstille genstande, der kan fremtræde eller bruges som våben. Dette har tidligere ikke har været tilgængeligt for den almene borger, da traditionel våbenproduktion er vanskelig. Eftersom denne nye metode, skaber en mulighed for privates produktion af skydevåben, er der muligvis behov for ændring af våbenlovgivningen, da overvågning af samtlige 3D-printerer i danmark er yderst urealitisk. En eventuel lovændring, vil dog også være problematisk, da det vil være svært at undgå at krænke grundlovssikrede friheder(7).

En alternativ løsning, som Aalborg universitet er ved at lave, er et modul som tillader printeren at genkende ulovlige 3D-schematics. Dette vil være en oplagt teknisk løsning på et juridisk problem.

\subsection{Lovgivningens utilstrækkelighed}
 
Disse eksempeler viser at den nuværende lovgivning ikke er tilstrækkeligt teknologisk fokuseret, til at indfatte et boom indenfor 3D-printere. Det er ikke utænkeligt at en kontroversiel højesteretsdom (eller sågar en byretsdom), vil medføre et endnu større pres på Christiansborg fra de danske domstole, der vil efterspørge juridisk klarhed, omkring digitale medier og nyere teknologi.
Det er i hvert fald tydeligt, at den nuværende lovgivning ikke er tilstrækkeligt dækkende, og indtil dette ændrer sig, vil det være meget vanskeligt at forudse 3D-printningsmarkedet.

\subsection{Mærkevare}

Når man har et produkt der adskiller sig fra andre tilsvarende produkter på markedet, kan det kaldes for en mærkevare. Typisk er disse produkter også tilknyttet en kendt brand.
Det behøves ikke nødvendigvis være et tøjmærke, en stol fra IKEA eller bil fra Skoda, er begge kategoriseret som en mærkevare. 
%overvej om overstående er en nødvendig forklaring.
Ofte kan et mærke genkendes med et tegn eller ikon, og ifølge varemærkeloven: § 2 "bestå af alle arter tegn, der er egnet til at adskille en virksomheds varer eller tjenesteydelser fra andre virksomheders, og som kan gengives grafisk". På den måde kan man genkende kvalitet eller prisniveau af et produkt. 
					%^^ mangler noget mere forklaring
Kan man opnå den samme form for mærkeidentitet, når produktet bliver printet af en 3D-printer hjemmefra?

Man kan sige at hvis producenten selv laver designet som printeren skal følge, må det printet produkt, være en del af mærket. Eksempelvis: IKEA sælger et schematic til en dugholder. Det bliver købt af en forbruger, som med sin egen printer laver en fysisk kopi. Ved at forbrugeren bruger IKEA’s opskrift til at lave kopien, kunne man argumentere for at det er under samme varemærke. 

NOVA designer en ny sweater til deres efterårskollektion og bestemmer sig for at udgive deres opskrift i deres butikker. Forbrugeren køber en pose med det garn der skal bruges dertil sammen med en strikkeopskrift. Forbrugeren kan da gå hjem og gå i gang med at strikke sweater efter opskriften, sidst påføres det medfølgende logo på sweateren. Forbrugeren får det materialer med som producenten selv vil bruge under produktionen. Mærkets materialekvalitet vil ofte blive set som at være en del af branded.  På den måde giver producenten udtryk for at produktet er en del af deres brand, selvom at den er produceret af en privat person. 
																																	
Gælder de samme præncipper for 3D-printning da det produceres mekanisk og er derfor ikke påvirket af brugens input. Når produkter bliver printet kan de være 100\% identiske til de varer der sælges i butikkerne, hvis der gøres bruge af de samme kvalitet, farver og materialer.
Producenter kunne vælge at fjerne deres garanti eller fortrydelsesret til et produkt. Hvis de mener at brugeren ikke har brugt et materiale de har anbefalet, eller brugt lav kvalitets printer.
Nogle brugere vil dog kun anderkende et produkt som at være en del af branded, hvis det reelt kommer fra producenten der ejer mærket. Det ser man blandt andet når man køber højkvalitets kopivarer.\autocite{_vi_????}
Det er ikke helt det samme, da dem der kopierer ofte har lavet deres design for at det skulle ligne originalenerne, hvor 3D-print kan laves ud fra producentens originale opskrift eller design.
%^^ omformuler kopi------------------------------------------------^^
