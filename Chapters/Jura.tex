%!TEX root = ../Master.tex

\section{Juridiske Problemstillinger}

Et af de største spørgsmål forbundet med 3D-printerens udvikling, er de juridiske komplikationer. Man har tidligere set store konflikter ved digitalisering af intellektuel ejendom, især indenfor film- og medieindustrien. Hvorvidt 3D-printeren vil medføre konflikter af samme proportion, er svært at sige, men det er sikkert, at der vil opstå usikkerhed omkring lovgivningen på området.

\subsection{Gældende Lovgivning}

Ifølge designlovens § 10, stk. 1 fremgår det tydeligt, at designret ikke kan udøves ved "handlinger, der foretages i private øjemed"\autocite{retsinformation.dk_designloven_2012}.

Ophavsretslovens § 12 giver lov til at fremstille enkelte eksemplarer til privat brug.
Dog påpeges det i § 12 stk. 4 af ophavsretsloven\autocite{retsinformation.dk_ophavsretsloven_2010}, at § 12 stk. 1 ikke giver ret til kopiering ved privat forbrug af bl.a. brugskunst, hvis der indgår benyttelse af fremmed medhjælp ved eksemplarfremstillingen.


Hvis en mulig fremtid indeholder, centraliseret forbrugertilgængelige printerstationer\autocite{bjorn_godske_dansk_2012}, vil det kræve en overvågning af printet materiale, såfremt printerstationer falder ind under stk. 4's henvisning af fremmed medhjælp. Det er ikke usandsynligt at en højesteretsdom, vil kunne kræve en slags overvågning af hvad der bliver printet, selvom det ikke er printerstationens ejer, der foretager den ulovlige handling. Her tænkes der især på sagen "Telenor mod IFPI" (med flere)\autocite{domstol.dk_telenor_2010}, hvor Telenor blev pålagt at hindre adgang til Thepiratebay.org.

\subsection{Retspraksis}

Der findes pt. ingen højesteretsdomme som omtaler kopi af design i private og andre ikke-kommercielle hensigter\autocite{domstol.dk_domme_????}. Dette er forståeligt eftersom der ikke findes et stort behov for retspraksis indenfor dette emne, men med 3D-printere under voldsom udvikling, er det næsten kun et spørgsmål om tid. Det er tydeligt at den første sag om 3D-print, kommer til at have store konsekvenser for følgende sager, i samme stil som Telenor mod IFPI dommen, fik det for andre danske internetudbydere\autocite{casper_ulsoe_pirate_2010}.

Edb-programmer er sidestillet med litterære værker når det gælder ophavsretsloven\autocite{retsinformation.dk_ophavsretsloven_2010}, så det er nærtliggende at betragte 3D-schematics, og andre printerfiler, som litterære værker, med dertilhørende ophavsret. Men eftersom en 3D-schematic, blot er en slags opskrift, er det derfor kun opskriften og ikke selve produktet som er beskyttet. Dette kan medføre et senarie, hvor en skanning, af et ophavsretsbeskyttet design bliver lagt ud til fri benyttelse, hvorefter det vil være lovligt at 3D-printe fra denne skanning til privat forbrug.

\subsection{Våbenloven}

Et andet vigtigt juridisk aspekt, er printning af genstande, der er ulovlige i sig selv, det mest åbenlyse eksempel værende våben.

Flere funktionsdygtige våben er blevet fremstillet ved hjælp af en privatejet 3D-printer. To eksempler på våben, som er blevet printet og affyret, er AR-15 (1), en semiautomatisk riffel, og The Liberator(2), en håndpistol. 

Schematic til The Liberator kan downloades fra nettet og printes på de fleste 3D-printere. Dette stiller krav til den danske lovgivning, og dens nuværende håndtering af våbenproduktion.

I USA har Cody Wilson(3) startet organisationen 'Defence Distributed'(4), som har til formål at dele opensource 3D-schematics til våben. Organisationen henviser til den amerikanske forfatnings stillingtagen til civilles våbenret(5). Selvom amerikanske myndigheder har forsøgt at fjerne 3D-schematic af The Liberator, er den allerede blevet spredt på internettet, og er tilgængelig for danske borgere.

Ifølge dansk våbenlovgivning(6), er det ulovligt at fremstille genstande, der kan fremtræde eller bruges som våben. Dette har tidligere ikke været tilgængeligt for den almene borger, da traditionel våbenproduktion er vanskelig. Eftersom denne nye metode, skaber en mulighed for civilles produktion af skydevåben, er der muligvis behov for ændring af våbenlovgivningen, da overvågning af samtlige 3D-printere i Danmark er yderst urealistisk. En eventuel lovændring, vil dog også være problematisk, eftersom det vil være svært at undgå krænkelse af grundlovssikrede friheder(7).

En alternativ løsning, som Aalborg universitet arbejder på, er et modul som tillader printeren at genkende ulovlige 3D-schematics. Dette vil være en oplagt teknisk løsning på et juridisk problem(8).

\subsection{Lovgivningens utilstrækkelighed}
 
Disse eksempler viser, at den nuværende lovgivning ikke er tilstrækkeligt teknologisk fokuseret til, at indfatte et boom indenfor 3D-printning. Det er ikke utænkeligt at en kontroversiel højesteretsdom (eller sågar en byretsdom), vil medføre et endnu større pres på Christiansborg fra de danske domstole. De vil efterspørge juridisk klarhed, omkring digitale medier og nyere teknologi.
Det er i hvert fald tydeligt, at den nuværende lovgivning ikke er tilstrækkeligt dækkende, og indtil dette ændrer sig, vil det være meget vanskeligt at forudse 3D-printningsmarkedet.

