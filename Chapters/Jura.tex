\chapter{Jura}
Eksempler på en af de mange måder hvor på brugen af 3D print i Danmark og USA kan have betydning i samfundet, her under vil jeg kigge på printning af ulovlige ting, og den lovmessig aspekt af dette. Det er allerede et godt diskoteret emne på internettet, hvor indtil flere funktionsdygtig våben er blevet fremstillet ved hjælp at en 3D printer, som alle kan få adgang til. Her er to eks. på våben som er blevet printet og afprøvet (1) AR-15, som er en semiautomaisk rifel, og (2) the liberator, som er en hånd pistol, begge scematic kan downloades fra nettet og printes på enhver 3D printer, men herind kommer det lovmessige aspekt, for er det lovligt at printe disse våben?
I USA har Cody Wilson startet organisationen "Defence Distributed" som har til formål om dele open source 3D tegninger til våben, da den amerikanske forfating siger at "Eftersom en velordnet milits er nødvendig for en fri stats sikkerhed, må der ikke gøres indgreb i folkets ret til at besidde og bære våben". dervar det ikke ulovligt for Cody Wilson at fremstille denne tegning, men da han lagde den frit tilgendelig på nettet greb, regeringen ind, og fik tegningerne fjernet  