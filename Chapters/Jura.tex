\chapter{Jura-printning af våben}
Herefter følger et eksempel på en af de mange måder, hvorpå brugen af 3D-print i Danmark og USA, kan have betydning i samfundet. 

Der indgår et kig på printning af ulovlige genstande, hvilket allerede er et meget omdiskuteret emne. 

Flere funktionsdygtig våben er blevet fremstillet ved hjælp af en privatejet 3D-printer. To eksempler på våben, som er blevet printet og affyret, er AR-15 (1), en semiautomatisk riffel, og The Liberator(2), en håndpistol. 

Schematic til The Liberator kan downloades fra nettet og printes på de fleste 3D printer. Dette stiller krav til den danske lovgivning, og dens nuværende håndtering af produktion af våben.

I USA har Cody Wilson(3) startet organisationen 'Defence Distributed'(4), som har til formål at dele opensource 3D-schematics til våben. Organisationsen henviser til den amerikanske forfatnings stillingstagen til privates ret til våben(5). Selvom amerikanske myndigheder har forsøgt at fjerne 3D-schematic af The Liberator, er den allerede blevet spredt på internettet, og er tilgenlig for danske borgere.

ifølge dansk våbenlovgivning(6), er det ulovligt at fremstille genstande, der kan fremtræde eller bruges som våben. Dette har tidligere ikke har været tilgængeligt for den almene borger, da traditionel våbenproduktion er vanskelig. Eftersom denne nye metode, skaber en mulighed for privates produktion af skydevåben, er der muligvis behov for ændring af våbenlovgivningen, da overvågning af samtlige 3D-printerer i danmark er yderst urealitisk. En eventuel lovændring, vil dog også være problematisk, da det vil være svært at undgå at krænke grundlovssikrede friheder(7).

En alternativ løsning, som Aalborg universitet er ved at lave, er et modul, som tillader printeren at genkende ulovlige 3D-schematics. Dette vil være en oplagt teknisk løsning på et juridisk problem.