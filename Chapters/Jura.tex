\chapter{Jura-printning af våben}
Eksempler på en af de mange måder, hvor på brugen af 3D print i Danmark og USA kan have betydning i samfundet. Her under vil jeg kigge på printning af ulovlige ting, og den lovmessig aspekt af dette. Det er allerede et godt diskoteret emne på internettet, hvor indtil flere funktionsdygtig våben er blevet fremstillet ved hjælp at en 3D printer, som alle kan få adgang til. Her er to eksembler på våben som er blevet printet og afprøvet (1) AR-15, som er en semiautomaisk rifel, og (2) the liberator, som er en hånd pistol, scematic til the libarator kan downloades fra nettet og printes på enhver 3D printer, herunder kommer det lovmessige aspekt, for er det lovligt at printe disse våben?
I USA har Cody Wilson startet organisationen "Defence Distributed" som har til formål om dele open source 3D tegninger til våben, da den amerikanske forfating siger at "Eftersom en velordnet milits er nødvendig for en fri stats sikkerhed, må der ikke gøres indgreb i folkets ret til at besidde og bære våben". derfor var det ikke ulovligt for Cody Wilson at fremstille denne tegning, men da han lagde den frit tilgendelig på nettet greb regeringen ind, og fik tegningen fjernet, men ikke før at 100.000 allerede havde downloaded tegningen.
ifølge den Dansk våbenlov givning, "Bekendtgørelse af lov om våben og eksplosivstoffer", 
Paragraf: 1. citat "Det er forbudt uden tilladelse af justitsministeren eller den, han dertil bemyndiger, at indføre eller fremstille, herunder samle:" 
- 1. Skydevåben samt genstande, der fremtræder som skydevåben og som følge af konstruktionen eller det anvendte materiale kan ombygges hertil,
- 2. dele, der er specielt konstrueret eller modificeret til et skydevåben, og som er væsentlige for anvendelsen af våbnet, herunder aftagelige magasiner, bundstykker, baskyler, løb, låsestole, piber, rammer, slæder og tromler.
er det ulovligt, at fremstille genstande der kan fremtræde eller bruges som et våben. Problemet er dog bare at der ingen beskyttelse mod at folk printer våben, men Aalborg universitet er ved at lave en løsning hertil, ved at lave et modul, så printeren kan genkende tegningerne til et våben. Dette er dog kun gældende i Danmark, da som nævnt printede Cody Wilson sit eget våben, men fordi han har lov til at fremstille våben, gjorde han ikke noget ulovligt.