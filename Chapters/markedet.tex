%!TEX root = ../Master.tex
\section{Markedet} % (fold)

Der er 2 platforme, hvorpå almene slutbrugere kan tilgå 3D-printning. Brugeren kan printe selv eller leje en printer. Fælles for de to platforme, er den manglende kendskab til 3D-printning.





Vi har igennem de fleste kapitler, snakket om 3D-printeren som værende det nye internet. Men som med så mange andre opfindelser, er det ikke altid lige nemt for den almene dansker, at se deres behov for produktet. Et klassisk eksempel på dette var iPad'en, som flere steder blev kaldet en forvokset iPhone, der ikke kunne noget nyt. Nu har den fundet sin vej ind i mange danske hjem, og de fleste bliver brugt i stor stil. 
Vi er ikke i tvivl om hvorvidt 3D-printeren har samme potentiale, hvis end ikke et større et, men hvilke behov opfylder den hos slutbrugeren? En undersøgelse viser, at 71 procent ikke ved hvad en 3D-printer kan, og kun 6 procent af de anspurgte havde interesse i at eje en 3D-printer.
Men hvordan kan dette hænge sammen med eksperternes udtalelser omkring 'større end internettet'? \autocite{financial_times_3d_2012} Her minder det igen om iPad'en. iPad'en var ikke den første tablet på markedet, men hvis du spørger folk, vil de fleste nok opfatte den som den første. Hvorfor? fordi den blev solgt af markedstingsfolk. Det samme behov har 3D-printeren. Den almene slutbruger skal fortælles sit behov for 3D-printeren, og forklares hvor i sin hverdag at den passer ind.

Et stort behov for slutbrugerne er prisen. Vi har ikke foretaget markedsundersøgelser, men hvis folk skal give mere end tusind kroner, vil de gerne have en god idé om produktets muligheder. I skrivende stund er de fleste 3D-printere forholdsvis indviklet, og kræver ofte teknologisk snilde der overskrider det normale behov. Normale printerer er i forvejen for indviklede for mange mennesker, og hvis 3D-printerene skal kunne sælges, skal de gøres meget mere brugervenlige.

Det måske største problem 3D-printeren har, er dens indirekte behovsopfyldelse. Hvis man kigger på Maslow's behovspyramide \autocite{abraham_harold_maslow_theory_1943} opfylder printeren direkte, ikke mange behov for den almene slutbruger. Endnu engang trækkes der dog paraleller til iPad'en, men også til computeren og fjernsynet, der alle er medier for andre produktes evne til at opfylde behov. Dette er et meget vigtigt aspekt, og hvis 3D-printeren nogensinde skal nå ud til en almen slutbruger, så er det af allerhøjeste prioritet, at få taget hånd om dette.




\subsection{Print selv} % (fold)
\label{sub:print_selv}


For at brugeren kan printe 3D-objekter selv, skal brugeren investere i en 3D-printer. Den mest udbredte printer teknologi til hjemmeprintere, er FDM-baserede printere. En af de billigste præ-samlede 3D-printere er Printrbot Simple til 399 \$. \autocite{_assembled_????}

For at printe et fysisk objekt, skal man bruge et 3D-schematic som typisk er af typen STL. Disse 3D-schematics kan enten fabrikeres selv vha. CAD-programmer på computeren, eller kan hentes på online-markedspladser. Der fås både gratis og betalingsschematics. Eksempler på markedspladser er 3D Burrito \autocite{_3d_2013-1} og Thingiverse \autocite{_thingiverse_????}.


\paragraph{Problemer/mangler} % (fold)
\label{par:problemer_mangler}

Når man får fat i et 3D-schematic er der stadig en process af slicing og konfigurering af printeren, der skal gøres, før man kan begynde at printe. Denne process er i sin nuværende tilstand for kompliceret til at hr. og fru Jensen uden videre kan 3D-printe.

% paragraph problemer_mangler (end)



Problemer ved print selv:

\begin{itemize}
	\item Brugervenlighed
	\item pris på 3D-printer
	\item mærkevare/vareintegritet (indsat nedenunder, godt nok?)
\end{itemize}

\paragraph{Mærkevare}

Når man har et produkt der adskiller sig fra andre tilsvarende produkter på markedet, kan det kaldes for en mærkevare. Typisk er disse produkter også tilknyttet et kendt brand.
Det behøves ikke nødvendigvis være et tøjmærke, en stol fra IKEA eller bil fra Skoda, er begge kategoriseret som en mærkevare. 

Ofte kan et mærke genkendes med et tegn eller ikon, og ifølge varemærkeloven: § 2 "bestå af alle arter tegn, der er egnet til at adskille en virksomheds varer eller tjenesteydelser fra andre virksomheders, og som kan gengives grafisk". På den måde kan man genkende kvalitet eller prisniveau af et produkt. 
Kan man opnå den samme form for mærkeidentitet, når produktet bliver printet af en 3D-printer hjemmefra?

Man kan sige at hvis producenten selv laver designet som printeren skal følge, må det printet produkt, være en del af mærket. Eksempelvis: IKEA sælger et schematic til en dugholder. Det bliver købt af en forbruger, som med sin egen printer laver en fysisk kopi. Ved at forbrugeren bruger IKEA’s opskrift til at lave kopien, kunne man argumentere for at det er under samme varemærke. 

Hanne Falkenbergs\autocite{hanne_falkenberg_hanne_????} designer en ny sweater til deres efterårskollektion og bestemmer sig for at udgive deres opskrift i deres butikker. Forbrugeren køber en pose med det garn der skal bruges dertil sammen med en strikkeopskrift. Forbrugeren kan da gå hjem og gå i gang med at strikke sweater efter opskriften, sidst påføres det medfølgende logo på sweateren. Forbrugeren får det materialer med som producenten selv vil bruge under produktionen. Mærkets materialekvalitet vil ofte blive set som at være en del af branded.  På den måde giver producenten udtryk for at produktet er en del af deres brand, selvom at den er produceret af en privat person. 
																
Gælder de samme præncipper for 3D-printning da det produceres mekanisk og er derfor ikke påvirket af brugens input. Når produkter bliver printet kan de være 100\% identiske til de varer der sælges i butikkerne, hvis der gøres bruge af de samme kvalitet, farver og materialer.
Producenter kunne vælge at fjerne deres garanti eller fortrydelsesret til et produkt. Hvis de mener at brugeren ikke har brugt et materiale de har anbefalet, eller brugt lav kvalitets printer.
Nogle brugere vil dog kun anderkende et produkt som at være en del af branded, hvis det reelt kommer fra producenten der ejer mærket. Det ser man blandt andet når man køber højkvalitets kopivarer.\autocite{_vi_????}
Det er ikke helt det samme, da dem der kopierer ofte har lavet deres design for at det skulle ligne originalenerne, hvor 3D-print kan laves ud fra producentens originale opskrift eller design.


% subsection print_selv (end)

\subsection{Leje print} % (fold)
\label{sub:leje_print}

Hvad er leje print?? (kort)


problemer ved leje print:

\begin{itemize}
	\item Det er for dyrt (hvad grunder dette?)
	\item leveringstid
	\item henvisning til jura
\end{itemize}

% subsection leje_print (end)

\subsection{Nær fremtid} % (fold)
\label{sub:f_lles_problemer}

Hvad sker der i den nære fremtid (når patenterne udløber)



\paragraph{Patenter} % (fold)

Idet de fleste patenter omhandlende 3D-printning er ved at udløbe, kan teknologierne der bruges til 3D-printere for alvor udvikle sig. Sådan en udvikling kunne man f.eks. se da patentet på FDM printer-teknologien udløb i 2009. \autocite{manyika_disruptive_2013} Årene efter FDM patentet udløb, faldt prisen på printere baseret på teknologien fra mange tusinde dollars til omkring 300 dollars. \autocite{mims_3d_2013} Samme udvikling ventes for teknologien "laser sintering", hvilket ifølge \autocite{mims_3d_2013} vil gøre det muligt at 3D-printe i samme kvalitet som eksisterende plastik produkter, der er produceret v.h.a sprøjtestøbning.

% subsection f_lles_problemer (end)




\paragraph{Salg af 3D-printere}

Salget for 3D-printere er under stor udvikling \autocite{wohler_sales_2012}. Det estimeres at det globale salg af 3D-printere i 2025 vil vokse til 4 billiarder \$. \autocite[110]{manyika_disruptive_2013}





