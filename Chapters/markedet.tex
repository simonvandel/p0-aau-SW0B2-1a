%!TEX root = ../Master.tex
\newpage
\section{Markedet} % (fold)

Der findes 2 platforme, hvorpå almene slutbrugere kan tilgå 3D-printning. Brugeren kan printe selv eller leje en printer. Fælles for de to platforme, er den almene befolknings manglende kendskab til 3D-printning.





Vi har igennem de fleste kapitler, snakket om 3D-printeren som værende det nye internet. Men som med så mange andre opfindelser, er det ikke altid lige nemt for den almene dansker, at se deres behov for produktet. Et klassisk eksempel på dette var iPad'en, som flere steder blev kaldet en forvokset iPhone, der ikke kunne noget nyt. Nu har den fundet sin vej ind i mange danske hjem, og de fleste bliver brugt i stor stil. 
Vi er ikke i tvivl om hvorvidt 3D-printeren har samme potentiale, hvis end ikke et større et, men hvilke behov opfylder den hos slutbrugeren? En undersøgelse viser, at 71 procent ikke ved hvad en 3D-printer kan, og kun 6 procent af de anspurgte havde interesse i at eje en 3D-printer. \autocite{shane_taylor_key_2013}
Men hvordan kan dette hænge sammen med eksperternes udtalelser omkring 'større end internettet'? \autocite{financial_times_3d_2012} Her minder det igen om iPad'en. iPad'en var ikke den første tablet på markedet, men hvis du spørger folk, vil de fleste nok opfatte den som den første. Hvorfor? Fordi den blev solgt af marketingfolk. Det samme behov har 3D-printeren. Den almene slutbruger skal fortælles at de har behov for 3D-printeren, og forklares hvor i deres hverdag den passer ind.

Et stort behov for slutbrugerne er prisen. Vi har ikke foretaget markedsundersøgelser, men hvis folk skal give mere end tusind kroner, vil de gerne have en god idé om produktets muligheder. I skrivende stund er de fleste 3D-printere forholdsvis indviklede, og kræver ofte teknologisk snilde der overskrider det normale behov. Normale printere er i forvejen for indviklede for mange mennesker, og hvis 3D-printerene skal kunne sælges, skal de gøres meget mere brugervenlige.

Det måske største problem 3D-printeren har, er dens indirekte behovsopfyldelse. Hvis man kigger på Maslow's behovspyramide \autocite{abraham_harold_maslow_theory_1943} opfylder printeren direkte, ikke mange behov for den almene slutbruger. Endnu engang trækkes der dog paraleller til iPad'en, men også til computeren og fjernsynet, der alle er medier for andre produktes evne til at opfylde behov. Dette er et meget vigtigt aspekt, og hvis 3D-printeren nogensinde skal nå ud til en almen slutbruger, så er det af allerhøjeste prioritet, at få taget hånd om dette.




\subsection{Print selv} % (fold)
\label{sub:print_selv}


For at brugeren kan printe 3D-objekter selv, skal brugeren investere i en 3D-printer. Den mest udbredte printer teknologi til hjemmeprintere, er FDM-baserede printere. En af de billigste præ-samlede 3D-printere er Printrbot Simple til 399 \$. \autocite{_assembled_????} Hvis man vil have en meget brugervenlig 3D-printer, skal man give væsentligt mere. F.eks. Makerbot Replicator til 2199 \$. \autocite{_makerbot_????}

Når man vil printe et fysisk objekt, skal man bruge et 3D-schematic som typisk er af filtypen STL. Disse 3D-schematics kan enten fabrikeres selv vha. CAD-programmer på computeren, eller kan hentes på online-markedspladser. Der fås både gratis og betalingsschematics. Eksempler på markedspladser er 3D Burrito \autocite{_3d_2013-1} og Thingiverse \autocite{_thingiverse_????}.


\paragraph{Problemer} % (fold)
\label{par:problemer_mangler}

Når man får fat i et 3D-schematic er der stadig en process af slicing og konfigurering af printeren, der skal gøres, før man kan begynde at printe. Denne process er i sin nuværende tilstand for kompliceret til at hr. og fru Jensen uden videre kan 3D-printe. Mange gange kræver det flere forsøg at finde de optimale printerindstillinger for at printe et givent objekt. Hvis printerindstillingerne ikke er helt optimale, vil det fysiske objekt være skrøbeligt eller være af en for dårlig kvalitet. \autocite{_3d_2013}

Et andet problem er printtiden. Afhængig af størrelsen af det ønskede objekt og hvilken kvalitet printeren er indstillet til, kan det sagtens tage 11 timer at printe en terning på 5x5x5 cm. \autocite{_3d_2013}

Som nævnt tidligere er prisen på en 3D-printer stadig høj. Men som alt anden teknologi vil prisen falde i takt med teknologien forbedres.

% paragraph problemer_mangler (end)




\subsection{Leje print} % (fold)
\label{sub:leje_print}
Der findes en service hvor at et tredjepartsfirma printer produkter for den private. Det forgår ved at den private selv skaffer sig et schematic, enten ved at købe det eller ved selv at designe et. Derefter betaler man et firma for at de stiller deres printer til rådighed, hvor de så vil sørge for at produktet bliver printet og sendt til den private. Et eksempel på en sådan service er shapeways. \autocite{_shapeways_????}

Der er visse problemer ved at bruge sådan en service da der er en række ulemper i forhold til at købe et produkt hos en almindelig handler, såsom Fakta. Først og fremmest er der kvaliteten. En undersøgelse fra Adv Manuf Technol????? siger at det ikke er muligt at printe i en lige så god en kvalitet som man kan opnå ved traditionelle fremstillingsmetoder. Det sågar kun når der snakkes om at printe i metaller og ved brug af en SLS(Selective Laser Sintering printer) som anses at være det fineste og mest præcise indenfor markedet. Det kunne få nogle brugere til at købe deres produkt hos en traditionel forhandler i stedet for at få den leveret fra en printer, hvis det er at kvaliteten skal være meget høj.
Hvis man bor inde i en by kan man gå ned i sin butik og købe varen. Det betyder at man kan få sit ønskede produkt in sin varetægt samme dag man køber det. Det er ikke tilfældet hos de nuværende service yder som kan findes, da de alle levere deres udprintede produkter med posten.
Udover dette, kan det være svært at finde plads i sin pengepung til at betale for den levering og service som er obligatorisk ved denne fremstillingsmetode. Fordi at produktet ved denne producent er special fremstillet til kunden, vil det altid være dyre, da det ikke bliver masseproduceret. Ved masseproduktion kan en del af de omkostninger som ligges fælles ofte reduceres. En enkelt genstand til for eksempelet brætspil, vil blive masseproduceret hos en traditionel forhandler. Ved et tredjeparts udprintnings producent vil produkterne højst sandsynligvis kun bive enkelt produceret, eller i hvert fald kun i små mængder. Nogle designs kunne være lavet af privatpersonen hvilket reducere chancen for at der skal laves mere end en kopi af produktet.

\subsection{Fælles for de 2 platforme} % (fold)


\paragraph{Mærkevare}

Når man har et produkt der adskiller sig fra andre tilsvarende produkter på markedet, kan det kaldes for en mærkevare. Typisk er disse produkter også tilknyttet et kendt brand.
Det behøver ikke nødvendigvis være et tøjmærke - en stol fra IKEA eller bil fra Skoda er begge kategoriseret som en mærkevare. 

Ofte kan et mærke genkendes med et tegn eller ikon, og ifølge varemærkeloven: § 2 "bestå af alle arter tegn, der er egnet til at adskille en virksomheds varer eller tjenesteydelser fra andre virksomheders, og som kan gengives grafisk". \autocite{_varemaerkeloven_2012} På den måde kan man genkende kvalitet eller prisniveau af et produkt. 
Kan man opnå den samme form for mærkeidentitet, når produktet bliver printet af en 3D-printer hjemmefra?

Man kan sige at hvis producenten selv laver designet som printeren skal følge, må det printet produkt, være en del af mærket. Eksempelvis: IKEA sælger et schematic til en dugholder. Det bliver købt af en forbruger, som med sin egen printer laver en fysisk kopi. Ved at forbrugeren bruger IKEA’s opskrift til at lave kopien, kunne man argumentere for at det er under samme varemærke. 

Hanne Falkenbergs \autocite{hanne_falkenberg_hanne_????} designer en ny sweater til deres efterårskollektion og bestemmer sig for at udgive deres opskrift i deres butikker. Forbrugeren køber en pose med det garn der skal bruges dertil sammen med en strikkeopskrift. Forbrugeren kan da gå hjem og gå i gang med at strikke sweater efter opskriften, sidst påføres det medfølgende logo på sweateren. Forbrugeren får det materialer med som producenten selv vil bruge under produktionen. Mærkets materialekvalitet vil ofte blive set som at være en del af branded.  På den måde giver producenten udtryk for at produktet er en del af deres brand, selvom at den er produceret af en privat person. 
																
Gælder de samme principper for 3D-printning da det produceres mekanisk og er derfor ikke påvirket af brugens input. Når produkter bliver printet kan de være 100\% identiske til de varer der sælges i butikkerne, hvis der gøres bruge af de samme kvalitet, farver og materialer.
Producenter kunne vælge at fjerne deres garanti eller fortrydelsesret til et produkt. Hvis de mener at brugeren ikke har brugt et materiale de har anbefalet, eller brugt lav kvalitets printer.
Nogle brugere vil dog kun anerkende et produkt som at værende en del af branded, hvis det reelt kommer fra producenten der ejer mærket. Det ser man blandt andet når man køber højkvalitets kopivarer.\autocite{_vi_????}
Det er ikke helt det samme, da dem der kopierer ofte har lavet deres design for at det skulle ligne originalenerne, hvor 3D-print kan laves ud fra producentens originale opskrift eller design.

% subsection print_selv (end)

\paragraph{Patenter} % (fold)

Idet de fleste patenter omhandlende 3D-printning er ved at udløbe, kan teknologierne der bruges til 3D-printere for alvor udvikle sig. Sådan en udvikling kunne man f.eks. se da patentet på FDM printer-teknologien udløb i 2009. \autocite{manyika_disruptive_2013} Årene efter FDM patentet udløb, faldt prisen på printere baseret på teknologien fra mange tusinde dollars til omkring 300 dollars. \autocite{mims_3d_2013} Samme udvikling ventes for teknologien "laser sintering", hvilket ifølge \autocite{mims_3d_2013} vil gøre det muligt at 3D-printe i samme kvalitet som eksisterende plastik produkter, der er produceret v.h.a sprøjtestøbning.

% subsection f_lles_problemer (end)




\paragraph{Salg af 3D-printere}

Salget for 3D-printere er under stor udvikling \autocite{wohler_sales_2012}. Det estimeres at det globale salg af 3D-printere i 2025 vil vokse til 4 billiarder \$. \autocite[110]{manyika_disruptive_2013}





