%!TEX root = ../Master.tex
\section{Teknologiske muligheder} % (fold)

\subsection{Hvad kan man printe?} % (fold)


\subsection{Materialer} % (fold)
\label{sub:materialer}

% subsection materialer (end)

\subsection{Patenter} % (fold)
\label{sub:patenter}

Idet de fleste patenter omhandlende 3D-printning er ved at udløbe, kan teknologierne der bruges til 3D-printere for alvor udvikle sig. Sådan en udvikling kunne man f.eks. se da patentet på FDM printer-teknologien udløb i 2009. \autocite{manyika_disruptive_2013} Årene efter FDM patentet udløb, faldt prisen på printere baseret på teknologien fra mange tusinde dollars til omkring 300 dollars. \autocite{mims_3d_2013} Samme udvikling ventes for teknologien "laser sintering", hvilket ifølge \autocite{mims_3d_2013} vil gøre det muligt at 3D-printe i samme kvalitet som eksisterende plastik produkter, der er produceret v.h.a sprøjtestøbning.

% subsection patenter (end)

\subsection{Manglende brugervenlighed} % (fold)
\label{sub:manglende_brugervenlighed}

Lige nu, for at operere en 3D-printer, skal man bruge en mængde viden for at kunne bruge den uden større problemer. At udprinte et objekt er på nuværende tidspunkt mere kompliceret end bare at trykke ’print’ som man vil gøre i Microsoft Word. I Word kan man sagtens printe uden at vide det store om printeren. Så længe at den er koblet til og har papir/blæk skal man bare trykke på ’OK’ og så skal dokumentet nok komme frem. 
Ved en 3D-printer skal man vide hvilken plast man bør bruge, hvilken temperatur der er tilpas, hvilken hastighed man kan køre og hvilken slags fil den printer man har, er i stand til at modtage og printe fra.
Alt dette gør at en 3D-printer ikke er så brugervenlig og er derfor mindre attraktiv for den gængse bruger.



% subsection manglende_brugervenlighed (end)