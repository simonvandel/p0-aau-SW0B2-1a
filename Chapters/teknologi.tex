%!TEX root = ../Master.tex
\section{Teknologien} % (fold)

% \subsection{Hvordan Virker 3D-printning} % (fold)
En 3D-printer laver tredimensionelle computer modeller om til fysiske objekter skabt i f.eks. plastik, gips eller metal. Langt de fleste 3D-printere, printer i solid plastik som ABS - en plastik type som man kender fra bl.a. LEGO klodser. Der findes forskellige måder at printe på, men den mest almindelige er Additive Manufacturing, også kaldet lag på lag metoden. Additive Manufacturing betyder at der printes materiale, lag på lag, og derved bygger objekter fra bunden og op. 


\subsection{Printer typer og materialer} % (fold)
\label{sub:materialer}
FDM (Fused deposition modeling) printere er den mest almindelige printer til privat brug. Der er findes mange typer FDM-printerer, og mange forskellige typer plastik som bruges på disse printerer. De to førende plastiktyper til hobbyprinterer og semiprofessionelle 3D-Printerer er PLA og ABS.\autocite{3d-guide:_????} 

%hvad består en 3d printer af? hvilke dele ingår etc. (måske en blueprint tegning?)

FDM-printere fungerer ved at dysen varmes op til plastikens smeltepunkt, mellem 180 og 210 grader, for begge plastik typer. Derefter presses en tynd tråd af plastik ud gennem dysen, mens dysen flytter sig i 2 akser for at tegne det første lag, med den smeltede plastik. 
Når dysen har lagt det første lag plastik, sænkes planen som der printes på, for at dysen igen kan begynd at printe et nyt lag plastik. Denne process gentages ind til printet er færdigt. 
Afstand som planen sænkes med bestemmer kvaliteten af printet, i takt med at afstand forøges falder klvaliteten af printet og printtiden formindskes.

%3D-Printerer har oftest en blæser til at køle spidsen af dysen, så den flydende plastik størkner hurtigst muligt efter at have forladt dysen. Plastikens størknehastighed(er det et ord?) er også afgørende for hvor hurtigt en printer kan printe. 

DMLS-printerer (Direct metal laser sintering) fungerer på en anden metode end FDM-printere. DMSL-printerer bruger en laser til at bind metalpulver sammen. Et lag af metal pulver lægges ud på print planet, eller oven på et af de tideligere printet lag. Derefter bruges en laser til at en varmer et koncentreret område af printet op, og smelter dette punkt fast til laget under. På steder hvor der ikke skal printes metal, udfyldes der sand istedet til at understøtte, eventuel lag af metal som kan forkomme over dette punkt.\autocite{manyika_disruptive_2013}
Når den er færdig med at printe skal printet bages(teknisk term?), og derefter kan støttesandet typisk fjernes med vand eller lufttryk. printet derefter er færdigt.\autocite{manyika_disruptive_2013}
\newpage

%medtag evt. metode om SLA

%mangler en  afrundning. evt. hold dig til plastik printerer og måske kun de metoder som er fælles for printerer som kan ende i et hjem. FOKUS PÅ HJEMME PRINT! :D

% chapter nuv_rende_tilstand_af_3d_print (end)


% subsection materialer (end)

\subsection{3D data typer}
Når man snakker om filtyper og dataformater i forbindelse med 3D-printning, er der 2 væsentlige typer som man gør brug af. Den første er selve modellen som typisk vil være i form af en STL fil eller lignende. Men printeren kan ikke printe ud fra en model, Den skal i stedet modtage instrukser. Instrukserne genreres på en computer, ud fra modellen. Printerne kan modtage instrukserne fra computeren, ved brug af en direkte datalinje. (fx. USB) Der udover er det også muligt at indlæse instrukserne, fra et lokalt medie på printeren. (fx. SD-kort)
Til at genere instrukserne bruges en slicer, som oftest bruger G-koder som instrukser. G-koder angiver koordinatsæt som printeren skal køre hen til, og ved hvilken hastighed den skal printe med.
%!TEX root = ../Master.tex
\lstset{ %
language=C++,                % choose the language of the code
basicstyle=\footnotesize,       % the size of the fonts that are used for the code
numbers=left,                   % where to put the line-numbers
numberstyle=\footnotesize,      % the size of the fonts that are used for the line-numbers
stepnumber=1,                   % the step between two line-numbers. If it is 1 each line will be numbered
numbersep=5pt,                  % how far the line-numbers are from the code
backgroundcolor=\color{white},  % choose the background color. You must add \usepackage{color}
showspaces=false,               % show spaces adding particular underscores
showstringspaces=false,         % underline spaces within strings
showtabs=false,                 % show tabs within strings adding particular underscores
frame=single,           % adds a frame around the code
tabsize=2,          % sets default tabsize to 2 spaces
captionpos=b,           % sets the caption-position to bottom
breaklines=true,        % sets automatic line breaking
breakatwhitespace=false,    % sets if automatic breaks should only happen at whitespace
escapeinside={\%*}{*)}          % if you want to add a comment within your code
}

\paragraph{STL} % (fold)

Indenfor 3D-modeller findes der utallige filtyper, næsten alle CAD-programmer har deres egen. Men skulle man tale om en standart for 3D-modeller, så er STL den mest universelle filtype. Det er også den mest almindelige indenfor distribuering af modeller til 3D-printning. \autocite{makerbot}
STL filtypen bruges til at gemme informationer om et fysisk 3D-objekt. STL indeholder kun oplysninger om modellens omrids, og kan derfor ikke indeholde informationer såsom modellens farver. 
STL filer kan være skrevet i ASCII format. 
En sådan fil er bygget op af flade trekanter i et tredimensionelt rum. 
Hver trekant beskrives ved brug af normalvektoren og 3 punkter, (X,Y,Z) som udgør trekantens hjørnepunkter. Normalvektoren, som beskriver hvordan trekanten vender, består også af 3 koordinater (X,Y,Z). Alle punkter skal skrives som Floating points\autocite{stl}.
En trekant kaldes for en facet, og skal opstille på følgende måde:
\begin{lstlisting}
          facet normal Nx Ny Nz
            outer loop
              vertex X1 Y1 Z1
              vertex X2 X2 Z2
              vertex X3 Y3 Z3
            endloop
          endfacet
\end{lstlisting}
\newpage
Facet normal betyder: "begynd trekant og normalvektor med koordinaterne: Nx, Ny, Nz." Outer loop starter indramning af de 3 koordinater til trekantens hjørnepunkter. Hver hjørnepunkt beskrives på følgende måde: "vertex X Y Z". Punkterne afsluttes med endfacet, efterfulgt af endloop, for at afslutte trekanten. 
For at definere hvor filen starter og slutter, bruges syntaksen solid "navn"  til start af 3D-model, og endsolid til afslutning af 3D-modellen.
Eksempel på en 3D-model(Simpel firkant)\autocite{Stl_Eksempel}.
\begin{lstlisting}
        solid cube_corner
          facet normal 0.0 -1.0 0.0
            outer loop
              vertex 0.0 0.0 0.0
              vertex 1.0 0.0 0.0
              vertex 0.0 0.0 1.0
            endloop
          endfacet
          facet normal 0.0 0.0 -1.0
            outer loop
              vertex 0.0 0.0 0.0
              vertex 0.0 1.0 0.0
              vertex 1.0 0.0 0.0
            endloop
          endfacet
          facet normal 0.0 0.0 -1.0
            outer loop
              vertex 0.0 0.0 0.0
              vertex 0.0 0.0 1.0
              vertex 0.0 1.0 0.0
            endloop
          endfacet
          facet normal 0.577 0.577 0.577
            outer loop
              vertex 1.0 0.0 0.0
              vertex 0.0 1.0 0.0
              vertex 0.0 0.0 1.0
            endloop
          endfacet
        endsolid
\end{lstlisting}
De to store problemer ved STL filtypen er, at der ingen mulighed for rettighedsbeskyttelse er. Der er heller ikke nogen mulighed, for at tilføje farver til modellen.
\lstset{ %
language=C++,                % choose the language of the code
basicstyle=\footnotesize,       % the size of the fonts that are used for the code
numbers=left,                   % where to put the line-numbers
numberstyle=\footnotesize,      % the size of the fonts that are used for the line-numbers
stepnumber=1,                   % the step between two line-numbers. If it is 1 each line will be numbered
numbersep=5pt,                  % how far the line-numbers are from the code
backgroundcolor=\color{white},  % choose the background color. You must add \usepackage{color}
showspaces=false,               % show spaces adding particular underscores
showstringspaces=false,         % underline spaces within strings
showtabs=false,                 % show tabs within strings adding particular underscores
frame=single,           % adds a frame around the code
tabsize=2,          % sets default tabsize to 2 spaces
captionpos=b,           % sets the caption-position to bottom
breaklines=true,        % sets automatic line breaking
breakatwhitespace=false,    % sets if automatic breaks should only happen at whitespace
escapeinside={\%*}{*)}          % if you want to add a comment within your code
}
\textbf{G-koder}
G-koder er et sprog som CNC maskiner som fræsere, og 3D-printer benytter sig af. DeT er instrukser som beskriver en rute, som boret eller printerhovedet bevæge sig i\cite{gkode}. 
Der findes flere hundrede forskellige G-koder funktioner, men en til 3D-printning, benyttes kun en funktion. Den en funktion som printeren bruger heder G1. G1 fortæller printerne at den skal bevæge printerhovedet i en lineær strækning fra, den nuværende position til Det angivende koordinat, efter G1.
Så en typisk G-kode kunne se ud som følgende: \newline
G1 X0 Y0 Z20 E0
\newline
Denne kode fortæller printeren at den skal køre til følgende koordinat (X, Y, Z): 0,0,20, hvor E er mængden af tråd der samtidighed skal presse ud gennem dysen.
En parameter mere kan tilføjes til G-koden, F som beskriver hvilken hastighed som rute skal køres med. Er der ikke angivet en hastighed bruges samme hastighed som den tidligere kørte rute. 
Hver G-kode adskilles ved brug af White spacing.
Et Eksempel på G-Koder: (En simpel firkant):
\begin{lstlisting}
G1 F200 X0 Y0 Z0 E0
G1 F200 X100 Y0 Z0 E100
G1 F200 X100 Y100 Z0 E100
G1 F200 X0 Y100 Z0 E100
G1 F200 0 0 Z0 E100
\end{lstlisting}

%\subsection{Patenter} % (fold)
%\label{sub:patenter}



% subsection patenter (end)

%\subsection{Manglende brugervenlighed} % (fold)
%\label{sub:manglende_brugervenlighed}

%Lige nu, for at operere en 3D-printer, skal man bruge en mængde viden for at kunne bruge den uden større problemer. At udprinte et objekt er på nuværende tidspunkt mere kompliceret end bare at trykke ’print’ som man vil gøre i Microsoft Word. I Word kan man sagtens printe uden at vide det store om printeren. Så længe at den er koblet til og har papir/blæk skal man bare trykke på ’OK’ og så skal dokumentet nok komme frem. 
%Ved en 3D-printer skal man vide hvilken plast man bør bruge, hvilken temperatur der er tilpas, hvilken hastighed man kan køre og hvilken slags fil den printer man har, er i stand til at modtage og printe fra.
%Alt dette gør at en 3D-printer ikke er så brugervenlig og er derfor mindre attraktiv for den gængse bruger.



% subsection manglende_brugervenlighed (end)